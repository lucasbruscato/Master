%% ------------------------------------------------------------------------- %%
\chapter{Modelagem}
\label{cap:modelagem}

Utilizando-se da base de dados para modelagem descrita na seção \ref{sec:criacao_da_base_para_modelagem} - cujas variáveis explicativas foram descritas na seção \ref{sec:variaveis_explicativas} e a variável resposta descrita na seção \ref{sec:variavel_resposta} - foram aplicados modelos de aprendizado de máquina com o intuito de obter o melhor desempenho para predição da variável resposta utilizando as variáveis explicativas construídas.

Os métodos de aprendizado de máquina utilizados foram a Regressão Linear, o Random Forest e o XGBoost, sendo que a afinação dos parâmetros de tais modelos foi feita utilizando o método Randomized Search - uma vez que é tido como mais eficiente que outro método comumente utilizado, o Grid Search (ver (ref.)). A avaliação de desempenho destes modelos foi feita utilizando-se de duas métricas, a raiz do erro quadrático médio e as bandas de acerto.

\section{Análise Descritiva}
\label{sec:analise_descritiva}



(?).

\section{Randomized Search}
\label{sec:randomized_search}

(?).

\section{Regressão Linear}
\label{sec:regressao_linear}

(?).

\section{Random Forest}
\label{sec:random_forest}

(?).

\section{XGBoost}
\label{sec:xgboost}

(?).

%% ------------------------------------------------------------------------- %%