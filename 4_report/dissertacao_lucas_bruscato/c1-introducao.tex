%% ------------------------------------------------------------------------- %%
\chapter{Introdução}
\label{cap:introducao}

\section{Motivação}
\label{sec:motivacao}

A corrupção no Brasil é um problema antigo, complexo e com origem muitas vezes atribuída aos primórdios da colonização brasileira pela coroa portuguesa \citep{LeiteMacedo2017}. Durante o processo de colonização foram observados comportamentos presentes no relacionamento entre colonos e nativos que, posteriormente analisados por meio de correspondências históricas, foram classificados como corrupção \citep{LeiteMacedo2017}.

É de conhecimento internacional a magnitude da corrupção no Brasil assim como suas implicações, dentre elas, pode ser citada a desigualdade social \citep{Alves2018}, que por sua vez implica em criminalidade \citep{ResendeAndrade2018} e assim sucessivamente como uma cadeia de markov.

Com a democratização da internet e a evolução da ciência de dados de um modo geral, diversos estudos estatísticos envolvendo transparência política e estudos relacionados com causa e consequência da corrupção (\citep{FerrazFinan2008} e \citet{Ransom2013:MSc}) foram possibilitados. Desta forma, auxiliarão na identificação de corrupção e possivelmente em seu combate.

Em agosto de 2015 a Controladoria-Geral da União (CGU) iniciou um trabalho que engloba o "Programa de Fiscalização em Entes Federativos" \cite{CGU}, um novo método de controle que está sendo aplicado desde então na avaliação dos recursos públicos federais repassados a estados, municípios e Distrito Federal. As fiscalizações tiveram como objetivo avaliar a aplicação dos recursos federais repassados aos municípios pelos Ministérios da Educação, Saúde e Integração Nacional, por exemplo. A seleção dos municípios a serem auditados foi feita por meio de diversas normas definidas pelo CGU \cite{CGU} sendo que anualmente foram selecionados em média sessenta municípios.

O resultado de tal trabalho governamental - o relatório de auditoria - contém informações e conclusões dos auditores sobre cada município baseado nas informações coletadas, cuja forma segue estrita observância às normas de fiscalização aplicáveis ao Serviço Público Federal (técnicas de inspeção física e registros fotográficos, análise documental, realização de entrevistas e aplicação de questionários).

Tais relatórios possuem informações que - de uma forma automatizada por meio de técnicas de aprendizado de máquina - podem ser extraídas e interpretadas como indicadores de corrupção (desvio de verba governamental), deficiência econômica ou social.

\section{Objetivo}
\label{sec:objetivo}

O presente trabalho tem como objetivo aplicar técnicas de processamento de linguagem natural (NLP) em relatórios de auditoria governamentais e estender os resultados de tal aplicação sobre técnicas de predição de aprendizado de máquina.

\section{Contribuições}
\label{sec:contribuicoes}

A principal contribuição deste trabalho é obter uma proposta de modelo estatístico utilizando informações dos censos do IBGE para predição de informações extraídas de relatórios de auditoria. Para tanto, foi feita toda a extração destas informações, tanto os relatórios utilizados no trabalho de \citep{FerrazFinan2008} quanto informações do censo do IBGE utilizadas, parcialmente, no trabalho de \citep{Goldani2001} e também apresentados interpretações destes relatórios utilizando aprendizado de máquina por meio de n-gram e tf-idf.

Também foi apresentado nesta dissertação uma adição ao trabalho de \citep{FerrazFinan2008} exibindo interpretações automatizadas sobre os relatórios que também foram utilizados pelos mesmos, de tal forma que a informação extraída pode ser replicada e utilizada por trabalhos futuros.

\section{Estrutura do Trabalho}
\label{sec:estrutura_do_trabalho}

Este trabalho é organizado em (?) seções, descritas abaixo.

No segundo capítulo são abordados os aspectos tecnológicos utilizados no trabalho, assim como os padrões de programação escolhidos. A forma como foram coletadas e extraídas as informações de todos os relatórios produzidos pela CGU no "Programa de Fiscalização de Entes Federativos". E, por fim, a forma como foram coletadas e criadas as bases de dados com informações coletadas dos censos do IBGE realizados nos anos 2000 e 2010.

No terceiro capítulo são analisados os dados coletados no segundo capítulo, assim como a criação de um modelo para predição do mesmo. (?)

No quarto capítulo (?).

No quinto capítulo (?). %são apresentadas as conclusões e perspectivas para trabalhos futuros.

No apêndice, são apresentados os programas para obtenção dos dados e as estimativas dos modelos abordados neste trabalho.

%% ------------------------------------------------------------------------- %%