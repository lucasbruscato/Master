% Arquivo LaTeX de exemplo de dissertação/tese a ser apresentados à CPG do IME-USP
% 
% Versão 5: Sex Mar  9 18:05:40 BRT 2012
%
% Criação: Jesús P. Mena-Chalco
% Revisão: Fabio Kon e Paulo Feofiloff
%  
% Obs: Leia previamente o texto do arquivo README.txt

\documentclass[12pt,twoside,a4paper]{book}

% ------------------------------------------------------------------- %
% pacotes 
\usepackage{graphicx}
\usepackage{url}
\usepackage[T1]{fontenc} 
\usepackage[portuguese]{babel}
\usepackage[utf8]{inputenc}
\usepackage{float}
\usepackage{amsmath}
\usepackage{amsthm}
\usepackage{amsfonts}
\usepackage{algpseudocode}
\usepackage{amssymb}
\usepackage[pdftex]{graphicx}           % usamos arquivos pdf/png como figuras
\usepackage{setspace}                   % espaçamento flexível
\usepackage{indentfirst}                % indentação do primeiro parágrafo
\usepackage{makeidx}                    % índice remissivo
\usepackage[nottoc]{tocbibind}          % acrescentamos a bibliografia/indice/conteudo no Table of Contents
\usepackage{courier}                    % usa o Adobe Courier no lugar de Computer Modern Typewriter
\usepackage{type1cm}                    % fontes realmente escaláveis
\usepackage{listings}                   % para formatar código-fonte (ex. em Java)
\usepackage{titletoc}
\usepackage[fixlanguage]{babelbib}
\usepackage[font=small,format=plain,labelfont=bf,up,textfont=it,up]{caption}
\usepackage[usenames,svgnames,dvipsnames]{xcolor}
\usepackage[a4paper,top=3.0cm,bottom=2.0cm,left=3.0cm,right=2.0cm]{geometry} % margens
\usepackage[pdftex,plainpages=false,pdfpagelabels,pagebackref,colorlinks=true,citecolor=DarkGreen,linkcolor=NavyBlue,urlcolor=DarkRed,filecolor=green,bookmarksopen=true]{hyperref} % links coloridos
\usepackage[all]{hypcap} % soluciona o problema com o hyperref e capitulos
\usepackage[round,sort,nonamebreak]{natbib} % citação bibliográfica textual(plainnat-ime.bst)

\fontsize{60}{62}\usefont{OT1}{cmr}{m}{n}{\selectfont}

% ------------------------------------------------------------------- %
% Cabeçalhos similares ao TAOCP de Donald E. Knuth
\usepackage{fancyhdr}
\pagestyle{fancy}
\fancyhf{}
\renewcommand{\chaptermark}[1]{\markboth{\MakeUppercase{#1}}{}}
\renewcommand{\sectionmark}[1]{\markright{\MakeUppercase{#1}}{}}
\renewcommand{\headrulewidth}{0pt}

\allowdisplaybreaks

% ------------------------------------------------------------------- %
\graphicspath{{./figuras/}}             % caminho das figuras (recomendável)
\frenchspacing                          % arruma o espaço: id est (i.e.) e exempli gratia (e.g.) 
\urlstyle{same}                         % URL com o mesmo estilo do texto e não mono-spaced
\makeindex                              % para o índice remissivo
\raggedbottom                           % para não permitir espaços extra no texto
\fontsize{60}{62}\usefont{OT1}{cmr}{m}{n}{\selectfont}
\cleardoublepage
\normalsize

% ------------------------------------------------------------------- %
% Opções de listing usados para o código fonte
% Ref: http://en.wikibooks.org/wiki/LaTeX/Packages/Listings
\lstset{
    basicstyle=\ttfamily,           % the size of the fonts that are used for the code
    stepnumber=1,                   % the step between two line-numbers. If it's 1 each line will be numbered
    showspaces=false,               % show spaces adding particular underscores
    showstringspaces=false,         % underline spaces within strings
    showtabs=false,                 % show tabs within strings adding particular underscores
    framerule=0.6pt,
    tabsize=2,                      % sets default tabsize to 2 spaces
    captionpos=b,                   % sets the caption-position to bottom
    breaklines=true,                % sets automatic line breaking
    breakatwhitespace=false,        % sets if automatic breaks should only happen at whitespace
    frame=single,                   % adds a frame around the code
    rulecolor=\color{gray},         % if not set, the frame-color may be changed on line-breaks within not-black text (e.g. comments (green here))
    extendedchars=true,
    xleftmargin=10pt,
    xrightmargin=10pt,
    framexleftmargin=10pt,
    framexrightmargin=10pt
}

% Comandos customizados
\newcommand{\diag}{\operatorname{diag}}
\newcommand{\yi}{{\boldsymbol{y}_{(i)}}}
\newcommand{\bhat}{{\hat{\boldsymbol{\beta}}}}
\newcommand{\sigsq}{\sigma^2}
\newcommand{\tausq}{{\tau^2}}
\newcommand{\ubf}{{\boldsymbol{u}}}
\newcommand{\vbf}{{\boldsymbol{v}}}
\newcommand{\tbf}{{\boldsymbol{t}}}
\newcommand{\ybf}{{\boldsymbol{y}}}
\newcommand{\transp}{{\mathsf{T}}}
\newcommand{\betabf}{{\boldsymbol{\beta}}}
\newtheorem{thm}{Teorema}
\newtheorem{prop}{Proposição}[chapter]
% ------------------------------------------------------------------- %
% Corpo do texto
\begin{document}
\frontmatter 
% cabeçalho para as páginas das seções anteriores ao capítulo 1 (frontmatter)
\fancyhead[RO]{{\footnotesize\rightmark}\hspace{2em}\thepage}
\setcounter{tocdepth}{2}
\fancyhead[LE]{\thepage\hspace{2em}\footnotesize{\leftmark}}
\fancyhead[RE,LO]{}
\fancyhead[RO]{{\footnotesize\rightmark}\hspace{2em}\thepage}
\onehalfspacing  % espaçamento

% ------------------------------------------------------------------- %
% CAPA
% Nota: O título para as dissertações/teses do IME-USP devem caber em um 
% orifício de 10,7cm de largura x 6,0cm de altura que há na capa fornecida pela SPG.
\thispagestyle{empty}
\begin{center}
    \vspace*{2.3cm}
    
    \Large{\textbf{Predição de polaridade negativa\\
    em relatórios de auditoria\\
    utilizando dados socioeconômicos}}
    
    \vspace*{1.2cm}
    \Large{Lucas Peinado Bruscato}
    
    \vskip 2cm
    \textsc{
    Dissertação apresentada\\[-0.25cm] 
    ao\\[-0.25cm]
    Instituto de Matemática e Estatística\\[-0.25cm]
    da\\[-0.25cm]
    Universidade de São Paulo\\[-0.25cm]
    para\\[-0.25cm]
    obtenção do título\\[-0.25cm]
    de\\[-0.25cm]
    Mestre em Ciências}
    
    \vskip 1.5cm
    Programa: Estatística\\
    Orientador: Profa. Dra. Florencia Leonardi

    \vskip 1cm
    \normalsize{}
    
    \vskip 0.5cm
    \normalsize{São Paulo, fevereiro de 2020}
\end{center}


\newpage
\thispagestyle{empty}
\begin{center}
    \vspace*{2.3 cm}
    
    \textbf{\Large{Predição de polaridade negativa\\
    em relatórios de auditoria\\
    utilizando dados socioeconômicos}}\\
    
    \vspace*{2 cm}
\end{center}

\vskip 2cm

\begin{flushright}
    Esta é a versão original da dissertação/tese elaborada pelo\\
    candidato Lucas Peinado Bruscato, tal como\\
    submetida à Comissão Julgadora.
    
    \vskip 2cm

\end{flushright}

\vskip 4.2cm

% ------------------------------------------------------------------- %
% Agradecimentos:
% Se o candidato não quer fazer agradecimentos, deve simplesmente eliminar esta página 
\chapter*{Agradecimentos}

(?).

\pagenumbering{roman}

% ------------------------------------------------------------------- %
% Resumo
\chapter*{Resumo}

(?).

% ------------------------------------------------------------------- %
% Abstract
\chapter*{Abstract}

(?).

% ------------------------------------------------------------------- %
% Sumário
\tableofcontents    % imprime o sumário

% ------------------------------------------------------------------- %
% Listas de figuras e tabelas criadas automaticamente
\listoffigures            
\listoftables            

% ------------------------------------------------------------------- %
% Capítulos do trabalho
\mainmatter

% cabeçalho para as páginas de todos os capítulos
\fancyhead[RE,LO]{\thesection}

\singlespacing              % espaçamento simples
%\onehalfspacing            % espaçamento um e meio

\input c1-introducao % associado ao arquivo: 'c1-introducao.tex'
\input c2-contexto % associado ao arquivo: 'c2-contexto.tex'
\input c3-modelagem % associado ao arquivo: 'c3-modelagem.tex'
\input c4-conclusao % associado ao arquivo: 'c4-conclusao.tex'

% cabeçalho para os apêndices
\renewcommand{\chaptermark}[1]{\markboth{\MakeUppercase{\appendixname\ \thechapter}} {\MakeUppercase{#1}} }
\fancyhead[RE,LO]{}
\appendix

\chapter{Códigos em \tt{Python}}
\label{ape:codigos}

\hypertarget{import-reports}{%
\section{Import reports}\label{import-reports}}

\begin{lstlisting}[language=Python]
from selenium import webdriver
from selenium.webdriver.support.ui import Select
from selenium.webdriver.common.keys import Keys
from selenium.webdriver.firefox.firefox_profile import FirefoxProfile

import os
import time
import codecs
\end{lstlisting}

\hypertarget{set-configurations-only-for-edicoes_anteriores-others-are-manually}{%
\subsubsection{Set configurations (only for `edicoes\_anteriores',
others are
manually)}\label{set-configurations-only-for-edicoes_anteriores-others-are-manually}}

\begin{lstlisting}[language=Python]
# set driver for browser connection
DRIVER_BIN = "/Users/lucas.bruscato/Google Drive/Github-Lucas/master/0_drivers/geckodriver_mac"
# DRIVER_BIN = "C:\\Users\\lbruscato\\Dropbox\\Github-Lucas\\master\\drivers\\geckodriver_windows.exe"

# set initial folder
folder = 'edicoes_anteriores/<folder name here>'
\end{lstlisting}

\hypertarget{capture-search-dates-and-cities-names}{%
\subsubsection{Capture search dates and cities'
names}\label{capture-search-dates-and-cities-names}}

\begin{lstlisting}[language=Python]
# open link file for the chosen folder
link_file = codecs.open(os.getcwd() + "/" + folder + "/link.txt", 'r', "utf-8")
text_link_file = link_file.read().split('\n')

# create the initial and final date to search
initial_date = (text_link_file[1].split('-'))[0]
final_date = (text_link_file[1].split('-'))[1]

# create an empty list of cities
cities = []

# fill cities list
for i in range(2, len(text_link_file)):
    city = text_link_file[i].split('-')
    cities.append(city[1].strip())

print("initial_date = " + initial_date)
print("final_date = " + final_date)
print(cities)
\end{lstlisting}

\hypertarget{open-browser-and-download-all-cities-report}{%
\subsubsection{Open browser and download all cities'
report}\label{open-browser-and-download-all-cities-report}}

\begin{lstlisting}[language=Python]
# set browser preferences and profile (auto download)
profile = FirefoxProfile()
profile.set_preference('browser.helperApps.neverAsk.saveToDisk', "application/pdf,application/zip")
profile.set_preference('browser.download.folderList', 2)
profile.set_preference('browser.download.dir', os.getcwd() + "/" + folder)

# open browser
browser = webdriver.Firefox(executable_path = DRIVER_BIN, firefox_profile = profile)
browser.maximize_window()

for j in range(0, len(cities), 5):
    browser.get('https://auditoria.cgu.gov.br/')
    
    # fill fields
    Select(browser.find_element_by_id("linhaAtuacao")).select_by_visible_text('Fiscalização em Entes Federativos - Municípios')

    browser.find_element_by_id("de").send_keys(initialDate, Keys.TAB)
    browser.find_element_by_id("ate").send_keys(finalDate, Keys.TAB)
    
    for i in range(j, j + 5):
        if (i < len(cities)):
            browser.find_element_by_id("palavraChave").send_keys(cities[i], Keys.COMMAND, 'a')
            browser.find_element_by_id("palavraChave").send_keys(Keys.COMMAND, 'x')
            browser.find_element_by_id("token-input-municipios").send_keys(Keys.COMMAND, 'v')
            time.sleep(2)
            
            browser.find_element_by_id("token-input-municipios").send_keys(Keys.ENTER)
            time.sleep(1)
    
    # search and download files
    browser.find_element_by_id("btnPesquisar").click()
    time.sleep(2)
    
    browser.find_element_by_id("btnSelectAll").click()
    time.sleep(1)
    
    browser.find_element_by_id("btnBaixar").click()
    time.sleep(1)
\end{lstlisting}


\hypertarget{rename-folders}{%
\section{Rename Folders}\label{rename-folders}}

\begin{lstlisting}[language=Python]
import os
import os.path
import random
\end{lstlisting}

\hypertarget{create-list-of-folders-to-access-and-rename-files}{%
\subsubsection{Create list of folders to access and rename
files}\label{create-list-of-folders-to-access-and-rename-files}}

\begin{lstlisting}[language=Python]
folders = ["2000/education",
           "2000/family",
           "2000/fertility",
           "2000/social_indicator",
           "2000/work",
           "2010/education",
           "2010/family",
           "2010/fertility",
           "2010/social_indicator",
           "2010/work"]

state_acronym_and_name_list = [
    ["ac", "acre"],
    ["al", "alagoas"],
    ["ap", "amapa"],
    ["am", "amazonas"],
    ["ba", "bahia"],
    ["ce", "ceara"],
    ["df", "distrito_federal"],
    ["es", "espirito_santo"],
    ["go", "goias"],
    ["ma", "maranhao"],
    ["mt", "mato_grosso"],
    ["ms", "mato_grosso_do_sul"],
    ["mg", "minas_gerais"],
    ["pa", "para"],
    ["pb", "paraiba"],
    ["pr", "parana"],
    ["pe", "pernambuco"],
    ["pi", "piaui"],
    ["rj", "rio_de_janeiro"],
    ["rn", "rio_grande_do_norte"],
    ["rs", "rio_grande_do_sul"],
    ["ro", "rondonia"],
    ["rr", "roraima"],
    ["sc", "santa_catarina"],
    ["sp", "sao_paulo"],
    ["se", "sergipe"],
    ["to", "tocantins"]]
\end{lstlisting}

\hypertarget{rename-files-based-on-pattern}{%
\subsubsection{Rename files based on
pattern}\label{rename-files-based-on-pattern}}

\begin{lstlisting}[language=Python]
for folder in folders:
    print(folder)
    for inner_folder in os.listdir(folder):
        new_folder_name = inner_folder
        for acronym in state_acronym_and_name_list:
            if inner_folder == acronym[0]:
                new_folder_name = acronym[1]
                break
        new_folder_name = str.replace(new_folder_name, "_munic_xls", "")
        new_folder_name = str.replace(new_folder_name, "_xls", "")
        
        print(folder + "/" + inner_folder + " -> " + folder + "/" + new_folder_name)
        
        os.rename(folder + "/" + inner_folder,
                  folder + "/" + new_folder_name)
\end{lstlisting}

\input{apendice/99_extract_explanatory_features.tex}
\hypertarget{extract-explanatory-features}{%
\section{Extract Explanatory
Features Enem}\label{extract-explanatory-features}}

\begin{lstlisting}[language=Python]
import os
import shutil
import pandas as pd
\end{lstlisting}

\hypertarget{extract-features-from-enem-2000}{%
\subsubsection{Extract features from enem
2000}\label{extract-features-from-enem-2000}}

\begin{lstlisting}[language=Python]
columns_to_select = ['NO_MUNICIPIO_RESIDENCIA',
                     'SG_UF_RESIDENCIA',
                     'TP_PRESENCA',
                     'TP_STATUS_REDACAO',
                     'NU_NOTA_OBJETIVA',
                     'NU_NOTA_REDACAO']
\end{lstlisting}

\begin{lstlisting}[language=Python]
df = pd.read_csv('2000/Dados/MICRODADOS_ENEM_2000.csv',
                 sep=';',
                 usecols=columns_to_select)

df = df.query('TP_PRESENCA == 1 and TP_STATUS_REDACAO == "P"')
\end{lstlisting}

\begin{lstlisting}[language=Python]
df['city_state'] = (df.NO_MUNICIPIO_RESIDENCIA + '_' + df.SG_UF_RESIDENCIA).str.normalize('NFKD')\
                    .str.encode('ascii', errors='ignore')\
                    .str.decode('utf-8').str.lower()

df['enem_score'] = (df['NU_NOTA_OBJETIVA']+df['NU_NOTA_REDACAO'])/2
\end{lstlisting}

\begin{lstlisting}[language=Python]
df_grouped = df[['city_state', 'enem_score']].groupby('city_state').agg(['mean', 'std', 'median'])

df_grouped.columns = ['enem_score_mean', 'enem_score_std', 'enem_score_median']
\end{lstlisting}

\begin{lstlisting}[language=Python]
df_grouped.to_csv('2000/2000_enem_score_var_01.csv', sep=';')
\end{lstlisting}

\hypertarget{extract-features-from-enem-2010}{%
\subsubsection{Extract features from enem
2010}\label{extract-features-from-enem-2010}}

\begin{lstlisting}[language=Python]
columns_to_select = ['NO_MUNICIPIO_RESIDENCIA',
                     'SG_UF_RESIDENCIA',
                     'TP_PRESENCA_CN',
                     'TP_PRESENCA_CH',
                     'TP_PRESENCA_LC',
                     'TP_PRESENCA_MT',
                     'NU_NOTA_CN',
                     'NU_NOTA_CH',
                     'NU_NOTA_LC',
                     'NU_NOTA_MT',
                     'TP_STATUS_REDACAO',
                     'NU_NOTA_REDACAO']
\end{lstlisting}

\begin{lstlisting}[language=Python]
df = pd.read_csv('2010/Dados/MICRODADOS_ENEM_2010.csv',
                 sep=';',
                 usecols=columns_to_select)

df = df.query('TP_PRESENCA_CN == 1 and TP_PRESENCA_CH == 1 and TP_PRESENCA_LC == 1 and TP_PRESENCA_MT == 1 and TP_STATUS_REDACAO == "P"')
\end{lstlisting}

\begin{lstlisting}[language=Python]
df['city_state'] = (df.NO_MUNICIPIO_RESIDENCIA + '_' + df.SG_UF_RESIDENCIA).str.normalize('NFKD')\
                    .str.encode('ascii', errors='ignore')\
                    .str.decode('utf-8').str.lower()

df['enem_score'] = (df['NU_NOTA_CN']+df['NU_NOTA_CH']+df['NU_NOTA_LC']+df['NU_NOTA_MT']+df['NU_NOTA_REDACAO'])/5
\end{lstlisting}

\begin{lstlisting}[language=Python]
df_grouped = df[['city_state', 'enem_score']].groupby('city_state').agg(['mean', 'std', 'median'])

df_grouped.columns = ['enem_score_mean', 'enem_score_std', 'enem_score_median']
\end{lstlisting}

\begin{lstlisting}[language=Python]
df_grouped.to_csv('2010/2010_enem_score_var_01.csv', sep=';')
\end{lstlisting}


\section{Create Initial SentiLex Database and Improve
it}\label{create-initial-sentilex-database-and-improve-it}

\begin{lstlisting}[language=Python]
import csv
import pandas as pd
import os
import os.path
import random
import PyPDF2
import unidecode
import pandas as pd
from collections import Counter
import csv
\end{lstlisting}

\subsubsection{Read SentiLex-PT02 and extract
polarity}\label{read-sentilex-pt02-and-extract-polarity}

\begin{lstlisting}[language=Python]
# read csv file
sentilex_database = pd.read_csv("SentiLex-flex-PT02.txt", header = None)
sentilex_database.columns = ["adjective", "description"]

# extract "polarity" from "description"
polarity = pd.DataFrame(sentilex_database.description.str.split('\;+').str[3].str.split('\=+').str[1])
sentilex_database = pd.concat([sentilex_database, polarity], axis = 1, join = 'outer')

# remove duplicates
sentilex_database = sentilex_database.iloc[:, [0, 2]].drop_duplicates()
sentilex_database.columns = ["adjective", "polarity"]

# select only polarities in [-1, 0, 1]
polarities = ["-1", "0", "1"]
sentilex_database = sentilex_database[sentilex_database.polarity.isin(polarities)]
\end{lstlisting}

\begin{lstlisting}[language=Python]
sentilex_database.head()
\end{lstlisting}

\begin{lstlisting}
<tr style="text-align: right;">
  <th></th>
  <th>adjective</th>
  <th>polarity</th>
</tr>
\end{lstlisting}

\begin{lstlisting}
<tr>
  <th>0</th>
  <td>à-vontade</td>
  <td>1</td>
</tr>
<tr>
  <th>1</th>
  <td>abafada</td>
  <td>-1</td>
</tr>
<tr>
  <th>2</th>
  <td>abafadas</td>
  <td>-1</td>
</tr>
<tr>
  <th>3</th>
  <td>abafado</td>
  <td>-1</td>
</tr>
<tr>
  <th>4</th>
  <td>abafados</td>
  <td>-1</td>
</tr>
\end{lstlisting}

\subsubsection{Save initial sentilex
database}\label{save-initial-sentilex-database}

\begin{lstlisting}[language=Python]
sentilex_database.to_csv("99_01_sentilex_database.csv",
                         sep = ';',
                         encoding = 'utf-8',
                         index = False)
\end{lstlisting}

\subsubsection{Define randomly reports for improving
SentiLex-PT02}\label{define-randomly-reports-for-improving-sentilex-pt02}

\begin{lstlisting}[language=Python]
folders = ["ciclo_3",
           "ciclo_4",
           "ciclo_5",
           "edicoes_anteriores/sorteio_34",
           "edicoes_anteriores/sorteio_35",
           "edicoes_anteriores/sorteio_36",
           "edicoes_anteriores/sorteio_37",
           "edicoes_anteriores/sorteio_38",
           "edicoes_anteriores/sorteio_39",
           "edicoes_anteriores/sorteio_40"]

file_names_and_paths = []

for folder in folders:
    directory = '../programa_de_fiscalizacao_em_entes_federativos/' + folder
    
    number_of_files = len([name for name in os.listdir(directory) if os.path.isfile(os.path.join(directory, name))]) - 3
    random.seed(7)
    random_file_number = int(random.uniform(0, number_of_files))
    
    file_name_and_path = directory + "/" + os.listdir(directory)[random_file_number]
    file_names_and_paths.append(file_name_and_path)
    
file_names_and_paths
\end{lstlisting}

\begin{lstlisting}
['../programa_de_fiscalizacao_em_entes_federativos/ciclo_3/9044-Putinga-RS.pdf',
 '../programa_de_fiscalizacao_em_entes_federativos/ciclo_4/10321-Uruguaiana-RS.pdf',
 '../programa_de_fiscalizacao_em_entes_federativos/ciclo_5/12311-Teresópolis-RJ.pdf',
 '../programa_de_fiscalizacao_em_entes_federativos/edicoes_anteriores/sorteio_34/1837-São Mateus-ES.pdf',
 '../programa_de_fiscalizacao_em_entes_federativos/edicoes_anteriores/sorteio_35/1906-Patrocínio-MG.pdf',
 '../programa_de_fiscalizacao_em_entes_federativos/edicoes_anteriores/sorteio_36/2483-Pontal do Paraná-PR.pdf',
 '../programa_de_fiscalizacao_em_entes_federativos/edicoes_anteriores/sorteio_37/2871-São José do Sul-RS.pdf',
 '../programa_de_fiscalizacao_em_entes_federativos/edicoes_anteriores/sorteio_38/2975-Presidente Kennedy-ES.pdf',
 '../programa_de_fiscalizacao_em_entes_federativos/edicoes_anteriores/sorteio_39/3179-São Domingos do Araguaia-PA.pdf',
 '../programa_de_fiscalizacao_em_entes_federativos/edicoes_anteriores/sorteio_40/3390-Goianésia do Pará-PA.pdf']
\end{lstlisting}

\subsubsection{Import reports, collect unique words and save words not
in
SentiLex-PT02}\label{import-reports-collect-unique-words-and-save-words-not-in-sentilex-pt02}

\begin{lstlisting}[language=Python]
print("List of reports read to improve SentiLex database")

words_without_polarity_full = pd.DataFrame(columns=['adjective', 'polarity'])

for file_number in range(0, len(file_names_and_paths)):
    
    file_name = file_names_and_paths[file_number]
    print(file_name)
    
    # create a pdf object
    file = open(file_name, 'rb')
    
    # create a pdf reader object
    file_reader = PyPDF2.PdfFileReader(file)

    # iterate all documents
    word_index = -1
    flag_in_a_word = 0
    words = []

    for i in range(file_reader.numPages):
        page = unidecode.unidecode(file_reader.getPage(i).extractText().lower())

        for j in range(len(page)):
            letter = page[j]

            if (not letter.isalpha()) and flag_in_a_word != 0:
                flag_in_a_word = 0
            elif letter.isalpha() and flag_in_a_word == 0:
                flag_in_a_word = 1
                word_index += 1
                words.append(letter)
            elif letter.isalpha() and flag_in_a_word != 0:
                words[word_index] += letter

    words_unique = pd.DataFrame(pd.DataFrame(words).iloc[:, 0].unique())
    words_unique.columns = ["adjective"]
    
    words_with_polarity = words_unique.merge(sentilex_database,
                                             left_on="adjective",
                                             right_on="adjective",
                                             how="left")
    
    words_without_polarity_full = pd.concat([words_without_polarity_full,
                                             words_with_polarity[words_with_polarity.polarity.isnull()]])


words_without_polarity_full = pd.DataFrame(words_without_polarity_full.adjective.unique())
words_without_polarity_full.columns = ['adjective']
words_without_polarity_full['polarity'] = ''

words_without_polarity_full.sort_values(by=['adjective'], inplace=True)

words_without_polarity_full.to_csv("improving_sentilex/99_create_improving_sentilex.csv",
                                   sep=';',
                                   encoding='utf-8',
                                   index=False)
\end{lstlisting}

\begin{lstlisting}
List of reports read to improve SentiLex database
../programa_de_fiscalizacao_em_entes_federativos/ciclo_3/9044-Putinga-RS.pdf
../programa_de_fiscalizacao_em_entes_federativos/ciclo_4/10321-Uruguaiana-RS.pdf
../programa_de_fiscalizacao_em_entes_federativos/ciclo_5/12311-Teresópolis-RJ.pdf
../programa_de_fiscalizacao_em_entes_federativos/edicoes_anteriores/sorteio_34/1837-São Mateus-ES.pdf
../programa_de_fiscalizacao_em_entes_federativos/edicoes_anteriores/sorteio_35/1906-Patrocínio-MG.pdf
../programa_de_fiscalizacao_em_entes_federativos/edicoes_anteriores/sorteio_36/2483-Pontal do Paraná-PR.pdf
../programa_de_fiscalizacao_em_entes_federativos/edicoes_anteriores/sorteio_37/2871-São José do Sul-RS.pdf
../programa_de_fiscalizacao_em_entes_federativos/edicoes_anteriores/sorteio_38/2975-Presidente Kennedy-ES.pdf
../programa_de_fiscalizacao_em_entes_federativos/edicoes_anteriores/sorteio_39/3179-São Domingos do Araguaia-PA.pdf
../programa_de_fiscalizacao_em_entes_federativos/edicoes_anteriores/sorteio_40/3390-Goianésia do Pará-PA.pdf
\end{lstlisting}


\section{Create Target Feature}\label{ape:create_target_feature}

\begin{lstlisting}[language=Python]
import PyPDF2
import unidecode
import pandas as pd
from collections import Counter
import csv
import os
import os.path
import random
import re
import datetime
\end{lstlisting}

\subsubsection{Create lists with all paths to all
files}\label{create-lists-with-all-paths-to-all-files}

\begin{lstlisting}[language=Python]
folders = ["ciclo_3",
           "ciclo_4",
           "ciclo_5",
           "edicoes_anteriores/sorteio_34",
           "edicoes_anteriores/sorteio_35",
           "edicoes_anteriores/sorteio_36",
           "edicoes_anteriores/sorteio_37",
           "edicoes_anteriores/sorteio_38",
           "edicoes_anteriores/sorteio_39",
           "edicoes_anteriores/sorteio_40"]

seq_folders = []
file_names = []
file_names_and_paths = []

for folder in folders:
    directory = '../programa_de_fiscalizacao_em_entes_federativos/' + folder
    
    number_of_files = len([name for name in os.listdir(directory) if os.path.isfile(os.path.join(directory, name))])
    
    for i in range(0, number_of_files):
        file_name_and_path = directory + "/" + os.listdir(directory)[i]
        if (".pdf" in file_name_and_path):
            seq_folders.append(folder)
            file_names.append(os.listdir(directory)[i])
            file_names_and_paths.append(file_name_and_path)

print('Example: \n' + file_names_and_paths[0:1][0])
\end{lstlisting}

\subsubsection{\texorpdfstring{Generate target feature for each report
(`read' and summarised the
polarity)}{Generate target feature for each report (read and summarised the polarity)}}\label{generate-target-feature-for-each-report-read-and-summarised-the-polarity}

\begin{lstlisting}[language=Python]
sentilex_database = pd.read_csv("../sentilex/99_01_sentilex_database.csv",
                                sep = ";")

sentilex_database.adjective = sentilex_database.adjective.str.normalize('NFKD').\
                                str.encode('ascii', errors='ignore').str.decode('utf-8')
\end{lstlisting}

\begin{lstlisting}[language=Python]
cities = pd.DataFrame()

print("List of reports read and summarised")

for file_number in range(0, len(file_names_and_paths)):
    folder = seq_folders[file_number]
    file_name = file_names[file_number]
    file_name_and_path = file_names_and_paths[file_number]
    print(str(datetime.datetime.now()) + ' ' + file_name_and_path)
    
    # read report using external library pdf miner and save in 'temp_report.txt'
    command_to_cmd = 'pdf2txt.py "' + file_name_and_path + '" > temp_report.txt'
    os.system(command_to_cmd)
    
    # read temporary file
    temporary_file = open('temp_report.txt', 'r')
    
    whole_text = ''
    
    for line in temporary_file:
        whole_text += line
    
    words = re.findall(r"[\w']+", unidecode.unidecode(re.sub('\d', ' ', whole_text).lower()))

    # create the frequencies
    words_freq = pd.DataFrame.from_dict(Counter(words), orient = 'index').reset_index()
    words_freq.columns = ['word', 'freq']
    words_freq['pct'] = words_freq['freq']/sum(words_freq.freq)

    # aggregate polarity
    words_freq_polarity = words_freq.merge(sentilex_database,
                                           left_on = "word",
                                           right_on = "adjective",
                                           how = "left").iloc[:, [0, 1, 2, 4]]
    
    words_freq_polarity_fill = words_freq_polarity.fillna(0)
    
    # summarise
    number_of_words = words_freq_polarity_fill.freq.sum()
    pct_pol_neg = words_freq_polarity_fill[words_freq_polarity_fill.polarity == -1].pct.sum()
    pct_pol_pos = words_freq_polarity_fill[words_freq_polarity_fill.polarity == 1].pct.sum()
    pct_pol_neu = words_freq_polarity_fill[words_freq_polarity_fill.polarity == 0].pct.sum()
    
    current_city = pd.DataFrame({"folder": folder,
                                 "file_name": file_name,
                                 "number_of_words": number_of_words,
                                 "pct_pol_neg": pct_pol_neg,
                                 "pct_pol_pos": pct_pol_pos,
                                 "pct_pol_neu": pct_pol_neu},
                                index = [0])
    
    cities = cities.append(current_city)
\end{lstlisting}

\begin{lstlisting}[language=Python]
os.remove("temp_report.txt")
\end{lstlisting}

\begin{lstlisting}[language=Python]
cities.to_csv("../target_feature/01_target_feature.csv",
              sep=';',
              encoding='utf-8',
              index=False)
\end{lstlisting}


\section{Data Processing}\label{data-processing}

\begin{lstlisting}[language=Python]
import pandas as pd
import os
\end{lstlisting}

\subsubsection{Read and handle target
feature}\label{read-and-handle-target-feature}

\begin{lstlisting}[language=Python]
target_feature = pd.read_csv('../target_feature/01_target_feature.csv',
                             sep=';')
\end{lstlisting}

\begin{lstlisting}[language=Python]
target_feature.head()
\end{lstlisting}

\begin{lstlisting}
<tr style="text-align: right;">
  <th></th>
  <th>folder</th>
  <th>file_name</th>
  <th>number_of_words</th>
  <th>pct_pol_neg</th>
  <th>pct_pol_pos</th>
  <th>pct_pol_neu</th>
</tr>
\end{lstlisting}

\begin{lstlisting}
<tr>
  <th>0</th>
  <td>ciclo_3</td>
  <td>8998-Santo Antônio de Jesus-BA.pdf</td>
  <td>45543</td>
  <td>0.015063</td>
  <td>0.032302</td>
  <td>0.954087</td>
</tr>
<tr>
  <th>1</th>
  <td>ciclo_3</td>
  <td>9024-Ulianópolis-PA.pdf</td>
  <td>17432</td>
  <td>0.018945</td>
  <td>0.022160</td>
  <td>0.959642</td>
</tr>
<tr>
  <th>2</th>
  <td>ciclo_3</td>
  <td>9010-Aldeias Altas-MA.pdf</td>
  <td>59605</td>
  <td>0.022763</td>
  <td>0.024140</td>
  <td>0.954407</td>
</tr>
<tr>
  <th>3</th>
  <td>ciclo_3</td>
  <td>9034-Paraíba do Sul-RJ.pdf</td>
  <td>15486</td>
  <td>0.014342</td>
  <td>0.029007</td>
  <td>0.957103</td>
</tr>
<tr>
  <th>4</th>
  <td>ciclo_3</td>
  <td>9045-Governador Celso Ramos-SC.pdf</td>
  <td>5177</td>
  <td>0.011985</td>
  <td>0.025130</td>
  <td>0.963657</td>
</tr>
\end{lstlisting}

\begin{lstlisting}[language=Python]
target_feature['temp'] = target_feature['file_name'].str.replace('[0-9]|.pdf|-', ' ', regex=True)\
    .str.normalize('NFKD').str.encode('ascii', errors='ignore').str.decode('utf-8').str.lower().str.strip()
target_feature['city'] = target_feature['temp'].str[:-3]
target_feature['state'] = target_feature['temp'].str[-2:]
target_feature['city_state'] = target_feature['city'].map(str) + '_' + target_feature['state']

target_feature = target_feature.drop("temp", axis=1)
target_feature.head()
\end{lstlisting}

\begin{lstlisting}
<tr style="text-align: right;">
  <th></th>
  <th>folder</th>
  <th>file_name</th>
  <th>number_of_words</th>
  <th>pct_pol_neg</th>
  <th>pct_pol_pos</th>
  <th>pct_pol_neu</th>
  <th>city</th>
  <th>state</th>
  <th>city_state</th>
</tr>
\end{lstlisting}

\begin{lstlisting}
<tr>
  <th>0</th>
  <td>ciclo_3</td>
  <td>8998-Santo Antônio de Jesus-BA.pdf</td>
  <td>45543</td>
  <td>0.015063</td>
  <td>0.032302</td>
  <td>0.954087</td>
  <td>santo antonio de jesus</td>
  <td>ba</td>
  <td>santo antonio de jesus_ba</td>
</tr>
<tr>
  <th>1</th>
  <td>ciclo_3</td>
  <td>9024-Ulianópolis-PA.pdf</td>
  <td>17432</td>
  <td>0.018945</td>
  <td>0.022160</td>
  <td>0.959642</td>
  <td>ulianopolis</td>
  <td>pa</td>
  <td>ulianopolis_pa</td>
</tr>
<tr>
  <th>2</th>
  <td>ciclo_3</td>
  <td>9010-Aldeias Altas-MA.pdf</td>
  <td>59605</td>
  <td>0.022763</td>
  <td>0.024140</td>
  <td>0.954407</td>
  <td>aldeias altas</td>
  <td>ma</td>
  <td>aldeias altas_ma</td>
</tr>
<tr>
  <th>3</th>
  <td>ciclo_3</td>
  <td>9034-Paraíba do Sul-RJ.pdf</td>
  <td>15486</td>
  <td>0.014342</td>
  <td>0.029007</td>
  <td>0.957103</td>
  <td>paraiba do sul</td>
  <td>rj</td>
  <td>paraiba do sul_rj</td>
</tr>
<tr>
  <th>4</th>
  <td>ciclo_3</td>
  <td>9045-Governador Celso Ramos-SC.pdf</td>
  <td>5177</td>
  <td>0.011985</td>
  <td>0.025130</td>
  <td>0.963657</td>
  <td>governador celso ramos</td>
  <td>sc</td>
  <td>governador celso ramos_sc</td>
</tr>
\end{lstlisting}

\subsubsection{Read explanatory features: education, family, fertility
and work (2000 and
2010)}\label{read-explanatory-features-education-family-fertility-and-work-2000-and-2010}

\begin{lstlisting}[language=Python]
raw_dataset = target_feature
\end{lstlisting}

\begin{lstlisting}[language=Python]
state_name_to_acronym = pd.DataFrame({'full_state_name': 
                                      ['acre', 
                                       'alagoas', 
                                       'amapa', 
                                       'amazonas', 
                                       'bahia', 
                                       'ceara', 
                                       'distrito_federal', 
                                       'espirito_santo', 
                                       'goias', 
                                       'maranhao', 
                                       'mato_grosso', 
                                       'mato_grosso_do_sul', 
                                       'minas_gerais', 
                                       'para', 
                                       'paraiba', 
                                       'parana', 
                                       'pernambuco', 
                                       'piaui', 
                                       'rio_de_janeiro', 
                                       'rio_grande_do_norte', 
                                       'rio_grande_do_sul', 
                                       'rondonia', 
                                       'roraima', 
                                       'santa_catarina', 
                                       'sao_paulo', 
                                       'sergipe', 
                                       'tocantins'],
                                      'acronym': ['ac',
                                                  'al',
                                                  'ap',
                                                  'am',
                                                  'ba',
                                                  'ce',
                                                  'df',
                                                  'es',
                                                  'go',
                                                  'ma',
                                                  'mt',
                                                  'ms',
                                                  'mg',
                                                  'pa',
                                                  'pb',
                                                  'pr',
                                                  'pe',
                                                  'pi',
                                                  'rj',
                                                  'rn',
                                                  'rs',
                                                  'ro',
                                                  'rr',
                                                  'sc',
                                                  'sp',
                                                  'se',
                                                  'to']})

var_list = ['var_01',
            'var_02',
            'var_03']
\end{lstlisting}

\begin{lstlisting}[language=Python]
paths = ['../ibge_censo/2000/education',
         '../ibge_censo/2000/family',
         '../ibge_censo/2000/fertility',
         '../ibge_censo/2000/work',
         '../ibge_censo/2010/education',
         '../ibge_censo/2010/family',
         '../ibge_censo/2010/fertility',
         '../ibge_censo/2010/work']
\end{lstlisting}

\begin{lstlisting}[language=Python]
for path in paths:
    
    for var_name in var_list:
        full_temp = pd.DataFrame()
        
        for state in os.listdir(path):
            if not state.startswith('.'):
                state_acronym = state_name_to_acronym.loc[
                    state_name_to_acronym.full_state_name == state]['acronym'].values[0]
        
                for filename in os.listdir(path + '/' + state):
                    if not filename.startswith('.') and filename.endswith(var_name + '.csv'):
        
                        temp = pd.read_csv(path + '/' + state + '/' + filename)
                        temp['city_state'] = temp['city'].map(str) + '_' + state_acronym
                        
                        full_temp = pd.concat([full_temp, temp])
        
        if full_temp.shape[0] != 0:
            full_temp = full_temp.add_prefix(path.split("/")[2] + '_' + path.split("/")[3] + '_' + var_name + '_')
            column_to_join = path.split("/")[2] + '_' + path.split("/")[3] + '_' + var_name + '_city_state'
            
            raw_dataset = pd.merge(raw_dataset,
                                   full_temp.iloc[:,1:],
                                   left_on="city_state",
                                   right_on=column_to_join,
                                   how="left")
            
            raw_dataset = raw_dataset.drop(column_to_join, axis=1)
            
            print(path + ' [' + var_name + '] ')
        
\end{lstlisting}

\begin{lstlisting}
../ibge_censo/2000/education [var_01] 
../ibge_censo/2000/family [var_01] 
../ibge_censo/2000/family [var_02] 
../ibge_censo/2000/fertility [var_01] 
../ibge_censo/2000/fertility [var_02] 
../ibge_censo/2000/fertility [var_03] 
../ibge_censo/2000/work [var_01] 
../ibge_censo/2000/work [var_02] 
../ibge_censo/2010/education [var_01] 
../ibge_censo/2010/family [var_01] 
../ibge_censo/2010/family [var_02] 
../ibge_censo/2010/fertility [var_01] 
../ibge_censo/2010/fertility [var_02] 
../ibge_censo/2010/fertility [var_03] 
../ibge_censo/2010/work [var_01] 
../ibge_censo/2010/work [var_02] 
\end{lstlisting}

\subsubsection{Read explanatory feature: social indicator (not in same
pattern as
others)}\label{read-explanatory-feature-social-indicator-not-in-same-pattern-as-others}

\begin{lstlisting}[language=Python]
paths = ['../ibge_censo/2010/social_indicator']
\end{lstlisting}

\subsubsection{Changing city name due to city being known by two
different
names}\label{changing-city-name-due-to-city-being-known-by-two-different-names}

\begin{lstlisting}[language=Python]
raw_dataset.loc[raw_dataset.file_name=='3238-São Valério da Natividade-TO.pdf', 'city_state'] = 'sao valerio_to'
\end{lstlisting}

\begin{lstlisting}[language=Python]
for path in paths:
    
    for var_name in var_list:
        full_temp = pd.DataFrame()
        
        for state in os.listdir(path):
            if not state.startswith('.'):
                state_acronym = state_name_to_acronym.loc[
                    state_name_to_acronym.full_state_name == state]['acronym'].values[0]
                
                for filename in os.listdir(path + '/' + state):
                    if not filename.startswith('.') and filename.endswith(var_name + '.csv'):
                        
                        temp = pd.read_csv(path + '/' + state + '/' + filename)
                        temp['city_state'] = temp['city'].map(str) + '_' + state_acronym
                        
                        full_temp = pd.concat([full_temp, temp])
                        
        if full_temp.shape[0] != 0:
            full_temp = full_temp.add_prefix(path.split("/")[3] + '_' + var_name + '_')
            column_to_join = path.split("/")[3] + '_' + var_name + '_city_state'
            
            raw_dataset = pd.merge(raw_dataset,
                                   full_temp.iloc[:,1:],
                                   left_on="city_state",
                                   right_on=column_to_join,
                                   how="left")
            
            raw_dataset = raw_dataset.drop(column_to_join, axis=1)
            
            print(path + ' [' + var_name + '] ')
        
\end{lstlisting}

\begin{lstlisting}
../ibge_censo/2010/social_indicator [var_01] 
../ibge_censo/2010/social_indicator [var_02] 
../ibge_censo/2010/social_indicator [var_03] 
\end{lstlisting}

\begin{lstlisting}[language=Python]
for c in raw_dataset.columns:
    print(c)
\end{lstlisting}

\begin{lstlisting}
folder
file_name
number_of_words
pct_pol_neg
pct_pol_pos
pct_pol_neu
city
state
city_state
2000_education_var_01_quantity
2000_family_var_01_total
2000_family_var_01_adequada
2000_family_var_01_semi_adequada
2000_family_var_01_inadequada
2000_family_var_02_qt
2000_fertility_var_01_total
2000_fertility_var_01_has_children
2000_fertility_var_01_children_born
2000_fertility_var_01_children_borned_live
2000_fertility_var_01_children_borned_dead
2000_fertility_var_02_total
2000_fertility_var_02_married
2000_fertility_var_02_separated
2000_fertility_var_02_divorced
2000_fertility_var_02_widow
2000_fertility_var_02_single
2000_fertility_var_03_total
2000_work_var_01_total
2000_work_var_01_domestic_regular
2000_work_var_01_domestic_irregular
2000_work_var_01_other_regular
2000_work_var_01_military_and_gov
2000_work_var_01_other_irregular
2000_work_var_02_total
2000_work_var_02_regular
2000_work_var_02_military_and_gov
2000_work_var_02_irregular
2000_work_var_02_employers
2000_work_var_02_entrepreneur
2010_education_var_01_quantity
2010_family_var_01_total
2010_family_var_01_adequada
2010_family_var_01_semi_adequada
2010_family_var_01_inadequada
2010_family_var_02_qt
2010_fertility_var_01_total
2010_fertility_var_01_has_children
2010_fertility_var_01_children_born
2010_fertility_var_01_children_borned_live
2010_fertility_var_01_children_borned_dead
2010_fertility_var_02_total
2010_fertility_var_02_married
2010_fertility_var_02_separated
2010_fertility_var_02_divorced
2010_fertility_var_02_widow
2010_fertility_var_02_single
2010_fertility_var_03_total
2010_work_var_01_total
2010_work_var_01_main_regular
2010_work_var_01_main_irregular
2010_work_var_01_other_regular
2010_work_var_01_other_irregular
2010_work_var_02_total
2010_work_var_02_regular
2010_work_var_02_military_and_gov
2010_work_var_02_irregular
2010_work_var_02_entrepreneur
2010_work_var_02_employers
social_indicator_var_01_2000_total
social_indicator_var_01_2010_total
social_indicator_var_01_2000_15_to_24_years
social_indicator_var_01_2010_15_to_24_years
social_indicator_var_01_2000_25_to_59_years
social_indicator_var_01_2010_25_to_59_years
social_indicator_var_01_2000_60_to_more_years
social_indicator_var_01_2010_60_to_more_years
social_indicator_var_02_2000_suitable
social_indicator_var_02_2010_suitable
social_indicator_var_02_2000_semi_suitable
social_indicator_var_02_2010_semi_suitable
social_indicator_var_02_2000_inappropriate
social_indicator_var_02_2010_inappropriate
social_indicator_var_03_2000_responsable_illiterate
social_indicator_var_03_2010_responsable_illiterate
social_indicator_var_03_2000_inappropriate_residence
social_indicator_var_03_2010_inappropriate_residence
social_indicator_var_03_2000_responsable_illiterate_and_inappropriate_residence
social_indicator_var_03_2010_responsable_illiterate_and_inappropriate_residence
\end{lstlisting}

\begin{lstlisting}[language=Python]
raw_dataset.to_csv('02_01_raw_dataset.csv',
                   sep=';',
                   index=False)
\end{lstlisting}


\section{Data Processing}\label{data-processing}

\begin{lstlisting}[language=Python]
import pandas as pd
import numpy as np
\end{lstlisting}

\subsubsection{Feature engineering}\label{feature-engineering}

\begin{lstlisting}[language=Python]
full_dataset = pd.read_csv('02_01_raw_dataset.csv',
                           sep=';')
\end{lstlisting}

\begin{lstlisting}[language=Python]
full_dataset = full_dataset.replace('-', 0)

full_dataset.iloc[:, 9:] = full_dataset.iloc[:, 9:].apply(pd.to_numeric)
\end{lstlisting}

For all the features created using IBGE, divide them from the position
in 2000 by the position in 2010, with the following observations:
\textbf{if the feature is not a proportion, then divide the feature by
the population size of the year (2000 or 2010)}

\begin{lstlisting}[language=Python]
full_dataset['education_var_01_qt_pct'] = (full_dataset['2000_education_var_01_quantity'] / full_dataset['2000_family_var_02_qt']) / (full_dataset['2010_education_var_01_quantity'] / full_dataset['2010_family_var_02_qt'])

full_dataset['family_var_01_adequada_pct'] = (full_dataset['2000_family_var_01_adequada'] / full_dataset['2000_family_var_01_total']) / (full_dataset['2010_family_var_01_adequada'] / full_dataset['2010_family_var_01_total'])
full_dataset['family_var_01_semi_adequada_pct'] = (full_dataset['2000_family_var_01_semi_adequada'] / full_dataset['2000_family_var_01_total']) / (full_dataset['2010_family_var_01_semi_adequada'] / full_dataset['2010_family_var_01_total'])
full_dataset['family_var_01_inadequada_pct'] = (full_dataset['2000_family_var_01_inadequada'] / full_dataset['2000_family_var_01_total']) / (full_dataset['2010_family_var_01_inadequada'] / full_dataset['2010_family_var_01_total'])

full_dataset['fertility_var_01_has_children_pct'] = (full_dataset['2000_fertility_var_01_has_children'] / full_dataset['2000_fertility_var_01_total']) / (full_dataset['2010_fertility_var_01_has_children'] / full_dataset['2010_fertility_var_01_total'])
full_dataset['fertility_var_01_children_born_pct'] = (full_dataset['2000_fertility_var_01_children_born'] / full_dataset['2000_fertility_var_01_total']) / (full_dataset['2010_fertility_var_01_children_born'] / full_dataset['2010_fertility_var_01_total'])
full_dataset['fertility_var_01_children_borned_live_pct'] = (full_dataset['2000_fertility_var_01_children_borned_live'] / full_dataset['2000_fertility_var_01_total']) / (full_dataset['2010_fertility_var_01_children_borned_live'] / full_dataset['2010_fertility_var_01_total'])
full_dataset['fertility_var_01_children_borned_dead_pct'] = (full_dataset['2000_fertility_var_01_children_borned_dead'] / full_dataset['2000_fertility_var_01_total']) / (full_dataset['2010_fertility_var_01_children_borned_dead'] / full_dataset['2010_fertility_var_01_total'])

full_dataset['fertility_var_02_married_pct'] = (full_dataset['2000_fertility_var_02_married'] / full_dataset['2000_fertility_var_02_total']) / (full_dataset['2010_fertility_var_02_married'] / full_dataset['2010_fertility_var_02_total'])
full_dataset['fertility_var_02_separated_pct'] = (full_dataset['2000_fertility_var_02_separated'] / full_dataset['2000_fertility_var_02_total']) / (full_dataset['2010_fertility_var_02_separated'] / full_dataset['2010_fertility_var_02_total'])
full_dataset['fertility_var_02_divorced_pct'] = (full_dataset['2000_fertility_var_02_divorced'] / full_dataset['2000_fertility_var_02_total']) / (full_dataset['2010_fertility_var_02_divorced'] / full_dataset['2010_fertility_var_02_total'])
full_dataset['fertility_var_02_widow_pct'] = (full_dataset['2000_fertility_var_02_widow'] / full_dataset['2000_fertility_var_02_total']) / (full_dataset['2010_fertility_var_02_widow'] / full_dataset['2010_fertility_var_02_total'])
full_dataset['fertility_var_02_single_pct'] = (full_dataset['2000_fertility_var_02_single'] / full_dataset['2000_fertility_var_02_total']) / (full_dataset['2010_fertility_var_02_single'] / full_dataset['2010_fertility_var_02_total'])

full_dataset['fertility_var_03_total_pct'] = (full_dataset['2000_fertility_var_03_total'] / full_dataset['2000_family_var_02_qt']) / (full_dataset['2010_fertility_var_03_total'] / full_dataset['2010_family_var_02_qt'])

full_dataset['work_var_01_regular_pct'] = ((full_dataset['2000_work_var_01_domestic_regular'] + full_dataset['2000_work_var_01_other_regular'] + full_dataset['2000_work_var_01_military_and_gov']) / full_dataset['2000_work_var_01_total']) / ((full_dataset['2010_work_var_01_main_regular'] + full_dataset['2010_work_var_01_other_regular']) / full_dataset['2010_work_var_01_total'])
full_dataset['work_var_01_irregular_pct'] = ((full_dataset['2000_work_var_01_domestic_irregular'] + full_dataset['2000_work_var_01_other_irregular']) / full_dataset['2000_work_var_01_total']) / ((full_dataset['2010_work_var_01_main_irregular'] + full_dataset['2010_work_var_01_other_irregular']) / full_dataset['2010_work_var_01_total'])

full_dataset['social_indicator_var_01_15_to_24_years_pct'] = (full_dataset['social_indicator_var_01_2000_15_to_24_years'] / full_dataset['social_indicator_var_01_2010_15_to_24_years'])
full_dataset['social_indicator_var_01_25_to_59_years_pct'] = (full_dataset['social_indicator_var_01_2000_25_to_59_years'] / full_dataset['social_indicator_var_01_2010_25_to_59_years'])
full_dataset['social_indicator_var_01_60_to_more_years_pct'] = (full_dataset['social_indicator_var_01_2000_60_to_more_years'] / full_dataset['social_indicator_var_01_2010_60_to_more_years'])

full_dataset['social_indicator_var_02_suitable_pct'] = full_dataset['social_indicator_var_02_2000_suitable'] / full_dataset['social_indicator_var_02_2010_suitable']
full_dataset['social_indicator_var_02_semi_suitable_pct'] = full_dataset['social_indicator_var_02_2000_semi_suitable'] / full_dataset['social_indicator_var_02_2010_semi_suitable']
full_dataset['social_indicator_var_02_inappropriate_pct'] = full_dataset['social_indicator_var_02_2000_inappropriate'] / full_dataset['social_indicator_var_02_2010_inappropriate']

full_dataset['social_indicator_var_03_responsable_illiterate_pct'] = full_dataset['social_indicator_var_03_2000_responsable_illiterate'] / full_dataset['social_indicator_var_03_2010_responsable_illiterate']
full_dataset['social_indicator_var_03_inappropriate_residence_pct'] = full_dataset['social_indicator_var_03_2000_inappropriate_residence'] / full_dataset['social_indicator_var_03_2010_inappropriate_residence']
full_dataset['social_indicator_var_03_responsable_illiterate_and_inappropriate_residence_pct'] = full_dataset['social_indicator_var_03_2000_responsable_illiterate_and_inappropriate_residence'] / full_dataset['social_indicator_var_03_2010_responsable_illiterate_and_inappropriate_residence']
\end{lstlisting}

\subsubsection{Select only generated features and remove rows with NaN
values on
it}\label{select-only-generated-features-and-remove-rows-with-nan-values-on-it}

\begin{lstlisting}[language=Python]
modeling_dataset = full_dataset.iloc[:, np.r_[3:6, 7, 88:113]].dropna()
\end{lstlisting}

\subsubsection{\texorpdfstring{Remove `inf' value for all
columns}{Remove inf value for all columns}}\label{remove-inf-value-for-all-columns}

\begin{lstlisting}[language=Python]
def replace_inf_by_max(df, col_name):
    max_for_column = max(df.loc[df[col_name] != np.inf, col_name])
    df.loc[df[col_name] == np.inf, col_name] = max_for_column
    
    return df
\end{lstlisting}

\begin{lstlisting}[language=Python]
for col in modeling_dataset.columns:
    modeling_dataset = replace_inf_by_max(modeling_dataset, col)
\end{lstlisting}

\subsubsection{\texorpdfstring{One hot encoding for `state'
feature}{One hot encoding for state feature}}\label{one-hot-encoding-for-state-feature}

\begin{lstlisting}[language=Python]
modeling_dataset = pd.concat([
    modeling_dataset,
    pd.get_dummies(modeling_dataset['state'], prefix = 'state')
], axis = 1)

modeling_dataset = modeling_dataset.drop(columns=['state'])
\end{lstlisting}

\begin{lstlisting}[language=Python]
modeling_dataset.tail()
\end{lstlisting}

\begin{lstlisting}
<tr style="text-align: right;">
  <th></th>
  <th>pct_pol_neg</th>
  <th>pct_pol_pos</th>
  <th>pct_pol_neu</th>
  <th>education_var_01_qt_pct</th>
  <th>family_var_01_adequada_pct</th>
  <th>family_var_01_semi_adequada_pct</th>
  <th>family_var_01_inadequada_pct</th>
  <th>fertility_var_01_has_children_pct</th>
  <th>fertility_var_01_children_born_pct</th>
  <th>fertility_var_01_children_borned_live_pct</th>
  <th>...</th>
  <th>state_pr</th>
  <th>state_rj</th>
  <th>state_rn</th>
  <th>state_ro</th>
  <th>state_rr</th>
  <th>state_rs</th>
  <th>state_sc</th>
  <th>state_se</th>
  <th>state_sp</th>
  <th>state_to</th>
</tr>
\end{lstlisting}

\begin{lstlisting}
<tr>
  <th>592</th>
  <td>0.016456</td>
  <td>0.030705</td>
  <td>0.953637</td>
  <td>1.002093</td>
  <td>0.293259</td>
  <td>1.029561</td>
  <td>1.842016</td>
  <td>0.932309</td>
  <td>1.203601</td>
  <td>1.176372</td>
  <td>...</td>
  <td>0</td>
  <td>0</td>
  <td>0</td>
  <td>0</td>
  <td>0</td>
  <td>0</td>
  <td>0</td>
  <td>0</td>
  <td>0</td>
  <td>0</td>
</tr>
<tr>
  <th>593</th>
  <td>0.019006</td>
  <td>0.031450</td>
  <td>0.952185</td>
  <td>0.996090</td>
  <td>0.839330</td>
  <td>1.067334</td>
  <td>3.925634</td>
  <td>0.928454</td>
  <td>1.245205</td>
  <td>1.237379</td>
  <td>...</td>
  <td>0</td>
  <td>0</td>
  <td>0</td>
  <td>0</td>
  <td>0</td>
  <td>0</td>
  <td>0</td>
  <td>0</td>
  <td>0</td>
  <td>0</td>
</tr>
<tr>
  <th>594</th>
  <td>0.021571</td>
  <td>0.025940</td>
  <td>0.954154</td>
  <td>1.080277</td>
  <td>0.000000</td>
  <td>0.779659</td>
  <td>3.520802</td>
  <td>1.016201</td>
  <td>1.416888</td>
  <td>1.339411</td>
  <td>...</td>
  <td>0</td>
  <td>0</td>
  <td>0</td>
  <td>0</td>
  <td>0</td>
  <td>0</td>
  <td>0</td>
  <td>0</td>
  <td>0</td>
  <td>0</td>
</tr>
<tr>
  <th>595</th>
  <td>0.016244</td>
  <td>0.026293</td>
  <td>0.958579</td>
  <td>0.759617</td>
  <td>0.636611</td>
  <td>0.870724</td>
  <td>4.013250</td>
  <td>0.944821</td>
  <td>1.236316</td>
  <td>1.182111</td>
  <td>...</td>
  <td>1</td>
  <td>0</td>
  <td>0</td>
  <td>0</td>
  <td>0</td>
  <td>0</td>
  <td>0</td>
  <td>0</td>
  <td>0</td>
  <td>0</td>
</tr>
<tr>
  <th>596</th>
  <td>0.012584</td>
  <td>0.027566</td>
  <td>0.962011</td>
  <td>0.929166</td>
  <td>0.913296</td>
  <td>1.166338</td>
  <td>12.687498</td>
  <td>0.978698</td>
  <td>1.098397</td>
  <td>1.109735</td>
  <td>...</td>
  <td>0</td>
  <td>0</td>
  <td>0</td>
  <td>0</td>
  <td>0</td>
  <td>0</td>
  <td>0</td>
  <td>0</td>
  <td>1</td>
  <td>0</td>
</tr>
\end{lstlisting}

5 rows × 52 columns

\begin{lstlisting}[language=Python]
modeling_dataset.to_csv('02_02_modeling_dataset.csv',
                        sep=';',
                        index=False)
\end{lstlisting}


\section{Data Processing}\label{data-processing}

\begin{lstlisting}[language=Python]
import pandas as pd
\end{lstlisting}

\subsubsection{Splitting the datasets (training and
validation)}\label{splitting-the-datasets-training-and-validation}

\begin{lstlisting}[language=Python]
modeling_dataset = pd.read_csv('02_02_modeling_dataset.csv',
                               sep=';')
\end{lstlisting}

\begin{lstlisting}[language=Python]
training_dataset = modeling_dataset.sample(frac=0.75,
                                           random_state=7)
validation_dataset = modeling_dataset.drop(training_dataset.index)
\end{lstlisting}

\begin{lstlisting}[language=Python]
print('=== Number of rows === \n' +
      'Training: ' + str(len(training_dataset)) + '\n' +
      'Validation: ' + str(len(validation_dataset)))
\end{lstlisting}

\begin{lstlisting}
=== Number of rows === 
Training: 422
Validation: 141
\end{lstlisting}

\begin{lstlisting}[language=Python]
training_dataset.head()
\end{lstlisting}

\begin{lstlisting}
<tr style="text-align: right;">
  <th></th>
  <th>pct_pol_neg</th>
  <th>pct_pol_pos</th>
  <th>pct_pol_neu</th>
  <th>education_var_01_qt_pct</th>
  <th>family_var_01_adequada_pct</th>
  <th>family_var_01_semi_adequada_pct</th>
  <th>family_var_01_inadequada_pct</th>
  <th>fertility_var_01_has_children_pct</th>
  <th>fertility_var_01_children_born_pct</th>
  <th>fertility_var_01_children_borned_live_pct</th>
  <th>...</th>
  <th>state_pr</th>
  <th>state_rj</th>
  <th>state_rn</th>
  <th>state_ro</th>
  <th>state_rr</th>
  <th>state_rs</th>
  <th>state_sc</th>
  <th>state_se</th>
  <th>state_sp</th>
  <th>state_to</th>
</tr>
\end{lstlisting}

\begin{lstlisting}
<tr>
  <th>408</th>
  <td>0.017929</td>
  <td>0.026417</td>
  <td>0.958357</td>
  <td>1.021423</td>
  <td>0.054541</td>
  <td>1.033242</td>
  <td>9.985730</td>
  <td>0.989622</td>
  <td>1.090811</td>
  <td>1.082467</td>
  <td>...</td>
  <td>0</td>
  <td>0</td>
  <td>0</td>
  <td>0</td>
  <td>0</td>
  <td>1</td>
  <td>0</td>
  <td>0</td>
  <td>0</td>
  <td>0</td>
</tr>
<tr>
  <th>188</th>
  <td>0.013723</td>
  <td>0.025105</td>
  <td>0.964276</td>
  <td>1.058820</td>
  <td>1.178352</td>
  <td>0.638225</td>
  <td>24.733100</td>
  <td>1.010927</td>
  <td>1.133640</td>
  <td>1.122890</td>
  <td>...</td>
  <td>0</td>
  <td>0</td>
  <td>0</td>
  <td>0</td>
  <td>0</td>
  <td>0</td>
  <td>1</td>
  <td>0</td>
  <td>0</td>
  <td>0</td>
</tr>
<tr>
  <th>97</th>
  <td>0.012676</td>
  <td>0.025575</td>
  <td>0.963083</td>
  <td>0.991957</td>
  <td>0.641255</td>
  <td>1.170128</td>
  <td>9.192867</td>
  <td>0.999439</td>
  <td>1.147375</td>
  <td>1.140280</td>
  <td>...</td>
  <td>0</td>
  <td>0</td>
  <td>0</td>
  <td>0</td>
  <td>0</td>
  <td>1</td>
  <td>0</td>
  <td>0</td>
  <td>0</td>
  <td>0</td>
</tr>
<tr>
  <th>431</th>
  <td>0.021631</td>
  <td>0.030575</td>
  <td>0.949254</td>
  <td>0.970031</td>
  <td>0.876919</td>
  <td>1.228749</td>
  <td>2.848914</td>
  <td>0.962917</td>
  <td>1.129891</td>
  <td>1.123553</td>
  <td>...</td>
  <td>0</td>
  <td>0</td>
  <td>0</td>
  <td>0</td>
  <td>0</td>
  <td>0</td>
  <td>0</td>
  <td>0</td>
  <td>0</td>
  <td>0</td>
</tr>
<tr>
  <th>475</th>
  <td>0.014607</td>
  <td>0.031221</td>
  <td>0.955085</td>
  <td>0.974727</td>
  <td>2.738600</td>
  <td>0.943244</td>
  <td>2.845043</td>
  <td>0.977073</td>
  <td>1.233269</td>
  <td>1.185966</td>
  <td>...</td>
  <td>1</td>
  <td>0</td>
  <td>0</td>
  <td>0</td>
  <td>0</td>
  <td>0</td>
  <td>0</td>
  <td>0</td>
  <td>0</td>
  <td>0</td>
</tr>
\end{lstlisting}

5 rows × 52 columns

\begin{lstlisting}[language=Python]
training_dataset.to_csv('02_03_training_dataset.csv',
                        sep=';',
                        index=False)
\end{lstlisting}

\begin{lstlisting}[language=Python]
validation_dataset.to_csv('02_03_validation_dataset.csv',
                          sep=';',
                          index=False)
\end{lstlisting}



\label{ape:model_visualization}
\hypertarget{model-visualization}{%
\section{Model Visualization}\label{model-visualization}}

\begin{lstlisting}[language=Python]
import pandas as pd
import matplotlib.pyplot as plt
\end{lstlisting}

\begin{lstlisting}[language=Python]
training_dataset = pd.read_csv('../2_database/data_processing/02_03_training_dataset.csv',
                               sep=';')

validation_dataset = pd.read_csv('../2_database/data_processing/02_03_validation_dataset.csv',
                                 sep=';')
\end{lstlisting}

\begin{lstlisting}[language=Python]
df = pd.concat([training_dataset, validation_dataset])
\end{lstlisting}

\begin{lstlisting}[language=Python]
df.head()
\end{lstlisting}

\begin{lstlisting}[language=Python]
for col in df.columns:
    print(col)
\end{lstlisting}

\begin{lstlisting}[language=Python]
for i in range(0, len(df.columns)):
    plt.scatter(df.iloc[:, 0], df.iloc[:, i])
    
    plt.title(df.columns[i])
    plt.xlabel(df.columns[0])
    
    plt.savefig("img/two_by_two_scatter_plot/" + 
                str(i).rjust(2, '0') + "_" + df.columns[i] + ".png")
    plt.show()
\end{lstlisting}



\label{ape:model_training}
\hypertarget{model-training}{%
\section{Model Training}\label{model-training}}

\begin{lstlisting}[language=Python]
import pandas as pd
import matplotlib.pyplot as plt
from matplotlib.pyplot import figure
import numpy as np
import re
import datetime
from numpy import inf
from scipy.stats import randint as sp_randint
from scipy.stats import uniform as sp_randFloat
import pickle
import shap
import warnings
warnings.simplefilter('ignore')

from sklearn.model_selection import train_test_split
from sklearn.metrics import mean_squared_error
from sklearn.metrics import r2_score
from sklearn import model_selection
from sklearn.model_selection import RandomizedSearchCV

from sklearn.linear_model import LinearRegression
from sklearn.ensemble import RandomForestRegressor
from xgboost import XGBRegressor
\end{lstlisting}

\begin{lstlisting}[language=Python]
training_dataset = pd.read_csv('../2_database/data_processing/02_03_training_dataset.csv',
                               sep=';')

validation_dataset = pd.read_csv('../2_database/data_processing/02_03_validation_dataset.csv',
                                 sep=';')
\end{lstlisting}

\begin{lstlisting}[language=Python]
print("Training and Validation size: " + str(training_dataset.shape) + " / " + str(validation_dataset.shape))
\end{lstlisting}

\begin{lstlisting}
Training and Validation size: (420, 53) / (140, 53)
\end{lstlisting}

\begin{lstlisting}[language=Python]
training_dataset.head()
\end{lstlisting}

\begin{lstlisting}[language=Python]
for col in training_dataset.columns:
    print(col)
\end{lstlisting}

\begin{lstlisting}[language=Python]
array = training_dataset.values

X_training = array[:, 1:]
Y_training = array[:, 0]
\end{lstlisting}

\begin{lstlisting}[language=Python]
array = validation_dataset.values

X_validation = array[:, 1:]
Y_validation = array[:, 0]
\end{lstlisting}

\hypertarget{function-for-regression-evaluation}{%
\subsubsection{Function for regression
evaluation}\label{function-for-regression-evaluation}}

\begin{lstlisting}[language=Python]
def regression_evaluation(Y_training, y_training_pred, Y_validation, y_validation_pred):
    
    random_simulation = np.random.randint(int(np.quantile(Y_training, 0)*10), int(np.quantile(Y_training, 1)*10), size=len(Y_validation)) * 0.1
    null_simulation = [np.mean(Y_training)] * len(Y_validation)
    
    print(
        "RMSE training: " + str(np.round(np.sqrt(mean_squared_error(Y_training, y_training_pred)), 4)) + "\n" +
        "RMSE validation: " + str(np.round(np.sqrt(mean_squared_error(Y_validation, y_validation_pred)), 4)) + "\n" + 
        "RMSE validation random model: " + str(np.round(np.sqrt(mean_squared_error(Y_validation, random_simulation)), 4)) + "\n" +
        "RMSE validation null model: " + str(np.round(np.sqrt(mean_squared_error(Y_validation, null_simulation)), 4))
    )

    print("\n")
    
    delta = 0.15
    
    overestimate_training_rate = np.round(sum((y_training_pred > Y_training * (1 + delta)) == True)/len(Y_training), 4)
    underestimate_training_rate = np.round(sum((y_training_pred < Y_training * (1 - delta)) == True)/len(Y_training), 4)
    wellestimate_training_rate = np.round(1-(overestimate_training_rate + underestimate_training_rate), 4)
    
    overestimate_validation_rate = np.round(sum((y_validation_pred > Y_validation * (1 + delta)) == True)/len(Y_validation), 4)
    underestimate_validation_rate = np.round(sum((y_validation_pred < Y_validation * (1 - delta)) == True)/len(Y_validation), 4)
    wellestimate_validation_rate = np.round(1-(overestimate_validation_rate + underestimate_validation_rate), 4)
    
    overestimate_random_rate = np.round(sum((random_simulation > Y_validation * (1 + delta)) == True)/len(Y_validation), 4)
    underestimate_random_rate = np.round(sum((random_simulation < Y_validation * (1 - delta)) == True)/len(Y_validation), 4)
    wellestimate_random_rate = np.round(1-(overestimate_random_rate + underestimate_random_rate), 4)
    
    overestimate_null_rate = np.round(sum((null_simulation > Y_validation * (1 + delta)) == True)/len(Y_validation), 4)
    underestimate_null_rate = np.round(sum((null_simulation < Y_validation * (1 - delta)) == True)/len(Y_validation), 4)
    wellestimate_null_rate = np.round(1-(overestimate_null_rate + underestimate_null_rate), 4)
    
    print(
        "BANDS training (underestimate | well | overestimate): " + str(underestimate_training_rate) + " | " + str(wellestimate_training_rate) + " | " + str(overestimate_training_rate) + "\n" + 
        "BANDS validation (underestimate | well | overestimate): " + str(underestimate_validation_rate) + " | " + str(wellestimate_validation_rate) + " | " + str(overestimate_validation_rate) + "\n" + 
        "BANDS validation random model (underestimate | well | overestimate): " + str(underestimate_random_rate) + " | " + str(wellestimate_random_rate) + " | " + str(overestimate_random_rate) + "\n" + 
        "BANDS validation null model (underestimate | well | overestimate): " + str(underestimate_null_rate) + " | " + str(wellestimate_null_rate) + " | " + str(overestimate_null_rate)
    )
    
    figure(num=None, figsize=(8, 6), dpi=80, facecolor='w', edgecolor='k')
    
    plt.plot([0, 1], [0, 1], 'k-', zorder=1)
    plt.scatter(Y_training, y_training_pred, label='training', zorder=2)
    plt.scatter(Y_validation, y_validation_pred, label='validation', zorder=3)
    
    plt.ylabel("predicted")
    plt.xlabel("observed")
    plt.xlim([0, 0.75])
    plt.ylim([0, 0.75])
    
    plt.legend()
\end{lstlisting}

\hypertarget{function-for-printing-shapley-value-importance}{%
\subsubsection{Function for printing shapley value
importance}\label{function-for-printing-shapley-value-importance}}

\begin{lstlisting}[language=Python]
def print_shapley_value_importance(model, X_matrix, feature_names):
    explainer = shap.TreeExplainer(model)
    shap_values = explainer.shap_values(X_matrix)
    
    shap.summary_plot(shap_values,
                      X_matrix,
                      feature_names=feature_names)
\end{lstlisting}

\hypertarget{linear-regression}{%
\subsubsection{1 - Linear Regression}\label{linear-regression}}

\begin{lstlisting}[language=Python]
linear_regression_model = LinearRegression()
linear_regression_model.fit(X_training, Y_training)

y_training_pred = linear_regression_model.predict(X_training)
y_validation_pred = linear_regression_model.predict(X_validation)
\end{lstlisting}

\begin{lstlisting}[language=Python]
linear_regression_model = pickle.load(open('03_02_linear_regression_model.pickle', 'rb'))

y_training_pred = linear_regression_model.predict(X_training)
y_validation_pred = linear_regression_model.predict(X_validation)
\end{lstlisting}

\begin{lstlisting}[language=Python]
regression_evaluation(Y_training,
                      y_training_pred,
                      Y_validation,
                      y_validation_pred)
\end{lstlisting}

\begin{lstlisting}[language=Python]
pickle.dump(linear_regression_model, open('03_02_linear_regression_model.pickle', 'wb'))
\end{lstlisting}

\begin{lstlisting}[language=Python]
linear_regression_model = pickle.load(open('03_02_linear_regression_model.pickle', 'rb'))
\end{lstlisting}

\begin{lstlisting}[language=Python]
i=0

for column_name in training_dataset.columns[1:]:
    print(column_name + " : " + str(linear_regression_model.coef_[i]))
    i = i+1
\end{lstlisting}

\hypertarget{random-forest}{%
\subsubsection{2 - Random Forest}\label{random-forest}}

\begin{lstlisting}[language=Python]
random_forest_model = RandomForestRegressor(random_state=7)

params = {'max_depth': [2, 4, 6, 8, 10, 12, 14, 16, 18, 20],
          'max_features': sp_randint(5, len(training_dataset.columns)-1),
          'min_samples_split': sp_randFloat(0.1, 0.8),
          'min_samples_leaf': sp_randFloat(0.05, 0.4),
          'n_estimators': [10, 25, 50, 75, 100],
          'bootstrap': [True, False],
          'criterion': ['mse']}

rs_random_forest_model = RandomizedSearchCV(
    random_forest_model,
    param_distributions=params,
    n_iter=300,
    cv=5,
    iid=False,
    refit=True,
    random_state=7)

rs_random_forest_model.fit(X_training, Y_training)

y_training_pred = rs_random_forest_model.predict(X_training)
y_validation_pred = rs_random_forest_model.predict(X_validation)
\end{lstlisting}

\begin{lstlisting}[language=Python]
rs_random_forest_model = pickle.load(open('03_02_rs_random_forest_model.pickle', 'rb'))

y_training_pred = rs_random_forest_model.predict(X_training)
y_validation_pred = rs_random_forest_model.predict(X_validation)
\end{lstlisting}

\begin{lstlisting}[language=Python]
regression_evaluation(Y_training,
                      y_training_pred,
                      Y_validation,
                      y_validation_pred)
\end{lstlisting}

\begin{lstlisting}[language=Python]
pickle.dump(rs_random_forest_model, open('03_02_rs_random_forest_model.pickle', 'wb'))
\end{lstlisting}

\begin{lstlisting}[language=Python]
rs_random_forest_model = pickle.load(open('03_02_rs_random_forest_model.pickle', 'rb'))
\end{lstlisting}

\begin{lstlisting}[language=Python]
rs_random_forest_model.best_params_
\end{lstlisting}

\begin{lstlisting}[language=Python]
random_forest_model = RandomForestRegressor(
    bootstrap=False,
    criterion='mse',
    max_depth=10,
    max_features=13,
    min_samples_leaf=0.05170415106602393,
    min_samples_split=0.4774005510811298,
    n_estimators=75,
    random_state=7)

random_forest_model.fit(X_training, Y_training)
\end{lstlisting}

\begin{lstlisting}[language=Python]
print_shapley_value_importance(random_forest_model,
                               X_validation,
                               training_dataset.columns[1:])
\end{lstlisting}

\hypertarget{xgboost}{%
\subsubsection{3 - XGBoost}\label{xgboost}}

\begin{lstlisting}[language=Python]
xgboost_model = XGBRegressor(random_state=7)

params = {'silent': [False],
          'max_depth': [2, 4, 6, 8, 10, 12, 14, 16, 18, 20],
          'learning_rate': [0.001, 0.01, 0.1, 1],
          'colsample_bytree': [0.4, 0.5, 0.6, 0.7, 0.8, 0.9, 1.0],
          'colsample_bylevel': [0.4, 0.5, 0.6, 0.7, 0.8, 0.9, 1.0],
          'min_child_weight': [0.5, 1.0, 3.0, 5.0, 7.0, 10.0],
          'gamma': [0, 0.25, 0.5, 1.0],
          'reg_lambda': [0.1, 1.0, 5.0, 10.0, 50.0, 100.0],
          'n_estimators': [10, 25, 50, 75, 100],
          'eval_metric': ['rmse']}

rs_xgboost_model = RandomizedSearchCV(xgboost_model,
                                      param_distributions=params,
                                      n_iter=300,
                                      cv=5,
                                      iid=False,
                                      refit=True,
                                      random_state=7)

rs_xgboost_model.fit(X_training, Y_training)

y_training_pred = rs_xgboost_model.predict(X_training)
y_validation_pred = rs_xgboost_model.predict(X_validation)
\end{lstlisting}

\begin{lstlisting}[language=Python]
rs_xgboost_model = pickle.load(open('03_02_rs_xgboost_model.pickle', 'rb'))

y_training_pred = rs_xgboost_model.predict(X_training)
y_validation_pred = rs_xgboost_model.predict(X_validation)
\end{lstlisting}

\begin{lstlisting}[language=Python]
regression_evaluation(Y_training,
                      y_training_pred,
                      Y_validation,
                      y_validation_pred)
\end{lstlisting}

\begin{lstlisting}[language=Python]
pickle.dump(rs_xgboost_model, open('03_02_rs_xgboost_model.pickle', 'wb'))
\end{lstlisting}

\begin{lstlisting}[language=Python]
rs_xgboost_model = pickle.load(open('03_02_rs_xgboost_model.pickle', 'rb'))
\end{lstlisting}

\begin{lstlisting}[language=Python]
rs_xgboost_model.best_params_
\end{lstlisting}

\begin{lstlisting}[language=Python]
xgboost_model = XGBRegressor(silent=False,
                             reg_lambda=100.0,
                             n_estimators=50,
                             min_child_weight=7.0,
                             max_depth=12,
                             learning_rate=0.1,
                             gamma=0,
                             eval_metric='rmse',
                             colsample_bytree=0.4,
                             colsample_bylevel=0.7,
                             random_state=7)

xgboost_model.fit(X_training, Y_training)
\end{lstlisting}

\begin{lstlisting}[language=Python]
print_shapley_value_importance(xgboost_model,
                               X_validation,
                               training_dataset.columns[1:])
\end{lstlisting}


\chapter{Tabelas}
\label{ape:tabelas}
\begin{table}[h] 
\centering
\caption{Nome de todas as variáveis explicativas utilizadas}
\label{tab:cap2_todas_variaveis}
\begin{adjustbox}{height=8.0cm}
\begin{tabular}{c}
\mbox{Nome das variáveis} \\
\hline
\scriptsize \verb|education_var_01_quantity_pct| \\
\scriptsize \verb|family_var_01_suitable_pct| \\
\scriptsize \verb|family_var_01_semi_suitable_pct| \\
\scriptsize \verb|family_var_01_inappropriate_pct| \\
\scriptsize \verb|fertility_var_01_has_children_pct| \\
\scriptsize \verb|fertility_var_01_children_born_pct| \\
\scriptsize \verb|fertility_var_01_children_borned_live_pct| \\
\scriptsize \verb|fertility_var_01_children_borned_dead_pct| \\
\scriptsize \verb|fertility_var_02_married_pct| \\
\scriptsize \verb|fertility_var_02_separated_pct| \\
\scriptsize \verb|fertility_var_02_divorced_pct| \\
\scriptsize \verb|fertility_var_02_widow_pct| \\
\scriptsize \verb|fertility_var_02_single_pct| \\
\scriptsize \verb|fertility_var_03_total_pct| \\
\scriptsize \verb|work_var_01_regular_pct| \\
\scriptsize \verb|work_var_01_irregular_pct| \\
\scriptsize \verb|social_indicator_var_01_15_to_24_years_pct| \\
\scriptsize \verb|social_indicator_var_01_25_to_59_years_pct| \\
\scriptsize \verb|social_indicator_var_01_60_to_more_years_pct| \\
\scriptsize \verb|social_indicator_var_02_suitable_pct| \\
\scriptsize \verb|social_indicator_var_02_semi_suitable_pct| \\
\scriptsize \verb|social_indicator_var_02_inappropriate_pct| \\
\scriptsize \verb|social_indicator_var_03_responsable_illiterate_pct| \\
\scriptsize \verb|social_indicator_var_03_inappropriate_residence_pct| \\
\scriptsize \verb|social_indicator_var_03_responsable_illiterate_and_inappropriate_residence_pct| \\
\scriptsize \verb|enem_var_01_enem_score_mean_pct| \\
\scriptsize \verb|enem_var_01_enem_score_std_pct| \\
\scriptsize \verb|enem_var_01_enem_score_median_pct| \\
\scriptsize \verb|state_ac| \\
\scriptsize \verb|state_al| \\
\scriptsize \verb|state_ba| \\
\scriptsize \verb|state_ce| \\
\scriptsize \verb|state_es| \\
\scriptsize \verb|state_go| \\
\scriptsize \verb|state_ma| \\
\scriptsize \verb|state_mg| \\
\scriptsize \verb|state_ms| \\
\scriptsize \verb|state_mt| \\
\scriptsize \verb|state_pa| \\
\scriptsize \verb|state_pb| \\
\scriptsize \verb|state_pe| \\
\scriptsize \verb|state_pi| \\
\scriptsize \verb|state_pr| \\
\scriptsize \verb|state_rj| \\
\scriptsize \verb|state_rn| \\
\scriptsize \verb|state_ro| \\
\scriptsize \verb|state_rr| \\
\scriptsize \verb|state_rs| \\
\scriptsize \verb|state_sc| \\
\scriptsize \verb|state_se| \\
\scriptsize \verb|state_sp| \\
\scriptsize \verb|state_to| \\
\hline
\end{tabular}
\end{adjustbox}

Fonte: Elaborada pelo autor
\end{table}


\begin{table}[h] 
\centering
\caption{Métricas de percentual de polaridade}
\label{tab:cap2_analise_descritiva}
\begin{tabular}{ccccc}
Métrica & Negativa & Positiva & Neutra & Não Encontrada \\
\hline
mediana & 1.70\% & 2.79\% & 0.74\% & 94.91\% \\
mínimo & 0.36\% & 1.49\% & 0.18\% & 92.76\% \\
máximo & 4.54\% & 4.54\% & 1.79\% & 97.04\% \\
desvio-padrão & 0.37\% & 0.35\% & 0.18\% & 0.59\% \\
\hline
\end{tabular}

Fonte: Elaborada pelo autor
\end{table}


\begin{table}[h]
\centering
\caption{Estimativas da regressão linear}
\label{tab:cap3_estimativa_reg_lin}
\begin{adjustbox}{width=\textwidth}
\begin{tabular}{cc}
Variável ou Categoria & Estimativa \\
\hline
\verb|education_var_01_quantity_pct| &  0.06117 \\
\verb|family_var_01_suitable_pct|  & -0.00021 \\
\verb|family_var_01_semi_suitable_pct| &  0.00140 \\
\verb|family_var_01_inappropriate_pct|  & -0.00055 \\
\verb|fertility_var_01_has_children_pct| &  0.02855 \\
\verb|fertility_var_01_children_born_pct| &  0.25068 \\
\verb|fertility_var_01_children_borned_live_pct|  & -0.28298 \\
\verb|fertility_var_01_children_borned_dead_pct|  & -0.00617 \\
\verb|fertility_var_02_married_pct| &  0.03087 \\
\verb|fertility_var_02_separated_pct|  & -0.00328 \\
\verb|fertility_var_02_divorced_pct| &  0.00997 \\
\verb|fertility_var_02_widow_pct| &  0.01008 \\
\verb|fertility_var_02_single_pct| &  0.06214 \\
\verb|fertility_var_03_total_pct| &  0.16219 \\
\verb|work_var_01_regular_pct| &  0.03426 \\
\verb|work_var_01_irregular_pct| &  0.04609 \\
\verb|social_indicator_var_01_15_to_24_years_pct|  & -0.00144 \\
\verb|social_indicator_var_01_25_to_59_years_pct|  & -0.02038 \\
\verb|social_indicator_var_01_60_to_more_years_pct| &  0.00484 \\
\verb|social_indicator_var_02_suitable_pct|  & -0.00088 \\
\verb|social_indicator_var_02_semi_suitable_pct|  & -0.01605 \\
\verb|social_indicator_var_02_inappropriate_pct|  & -0.00055 \\
\verb|social_indicator_var_03_responsable_illiterate_pct| &  0.01571 \\
\verb|social_indicator_var_03_inappropriate_residence_pct|  & -0.00011 \\
\verb|social_indicator_var_03_responsable_illiterate_and_inappropriate_residence_pct| &  0.00063 \\
\verb|enem_var_01_enem_score_mean_pct| &  0.33489 \\
\verb|enem_var_01_enem_score_std_pct|  & -0.01912 \\
\verb|enem_var_01_enem_score_median_pct|  & -0.34682 \\
\verb|state_ac|  & -0.07552 \\
\verb|state_al| &  0.01843 \\
\verb|state_ba| &  0.01230 \\
\verb|state_ce|  & -0.00753 \\
\verb|state_es| &  0.01247 \\
\verb|state_go| &  0.03169 \\
\verb|state_ma| &  0.06556 \\
\verb|state_mg| &  0.04068 \\
\verb|state_ms|  & -0.00889 \\
\verb|state_mt| &  0.00559 \\
\verb|state_pa| &  0.00188 \\
\verb|state_pb| &  0.02969 \\
\verb|state_pe| &  0.00794 \\
\verb|state_pi|  & -0.00892 \\
\verb|state_pr|  & -0.00117 \\
\verb|state_rj|  & -0.00751 \\
\verb|state_rn| &  0.00427 \\
\verb|state_ro| &  0.01943 \\
\verb|state_rr|  & -0.02139 \\
\verb|state_rs|  & -0.02576 \\
\verb|state_sc|  & -0.01996 \\
\verb|state_se| &  0.01385 \\
\verb|state_sp|  & -0.02229 \\
\verb|state_to|  & -0.06484 \\
\hline
\end{tabular}

\end{adjustbox}

Fonte: Elaborada pelo autor
\end{table}


\chapter{Gráficos de Dispersão}
\label{ape:graficos_dispersao}

\label{ape:cap2_scatter_plot}

\graphicspath{ {./figuras/two_by_two_scatter_plot/} }

\includegraphics{00_pct_pol_neg_rel}
\includegraphics{01_education_var_01_quantity_pct}
\includegraphics{02_family_var_01_suitable_pct}
\includegraphics{03_family_var_01_semi_suitable_pct}
\includegraphics{04_family_var_01_inappropriate_pct}
\includegraphics{05_fertility_var_01_has_children_pct}
\includegraphics{06_fertility_var_01_children_born_pct}
\includegraphics{07_fertility_var_01_children_borned_live_pct}
\includegraphics{08_fertility_var_01_children_borned_dead_pct}
\includegraphics{09_fertility_var_02_married_pct}
\includegraphics{10_fertility_var_02_separated_pct}
\includegraphics{11_fertility_var_02_divorced_pct}
\includegraphics{12_fertility_var_02_widow_pct}
\includegraphics{13_fertility_var_02_single_pct}
\includegraphics{14_fertility_var_03_total_pct}
\includegraphics{15_work_var_01_regular_pct}
\includegraphics{16_work_var_01_irregular_pct}
\includegraphics{17_social_indicator_var_01_15_to_24_years_pct}
\includegraphics{18_social_indicator_var_01_25_to_59_years_pct}
\includegraphics{19_social_indicator_var_01_60_to_more_years_pct}
\includegraphics{20_social_indicator_var_02_suitable_pct}
\includegraphics{21_social_indicator_var_02_semi_suitable_pct}
\includegraphics{22_social_indicator_var_02_inappropriate_pct}
\includegraphics{23_social_indicator_var_03_responsable_illiterate_pct}
\includegraphics{24_social_indicator_var_03_inappropriate_residence_pct}
\includegraphics{25_social_indicator_var_03_responsable_illiterate_and_inappropriate_residence_pct}
\includegraphics{26_enem_var_01_enem_score_mean_pct}
\includegraphics{27_enem_var_01_enem_score_std_pct}
\includegraphics{28_enem_var_01_enem_score_median_pct}
\includegraphics{29_state_ac}
\includegraphics{30_state_al}
\includegraphics{31_state_ba}
\includegraphics{32_state_ce}
\includegraphics{33_state_es}
\includegraphics{34_state_go}
\includegraphics{35_state_ma}
\includegraphics{36_state_mg}
\includegraphics{37_state_ms}
\includegraphics{38_state_mt}
\includegraphics{39_state_pa}
\includegraphics{40_state_pb}
\includegraphics{41_state_pe}
\includegraphics{42_state_pi}
\includegraphics{43_state_pr}
\includegraphics{44_state_rj}
\includegraphics{45_state_rn}
\includegraphics{46_state_ro}
\includegraphics{47_state_rr}
\includegraphics{48_state_rs}
\includegraphics{49_state_sc}
\includegraphics{50_state_se}
\includegraphics{51_state_sp}
\includegraphics{52_state_to}


% ------------------------------------------------------------------- %
% Bibliografia
\backmatter \singlespacing   % espaçamento simples
\bibliographystyle{plainnat-ime} % citação bibliográfica textual
\bibliography{bibliografia}  % associado ao arquivo: 'bibliografia.bib'

\end{document}
