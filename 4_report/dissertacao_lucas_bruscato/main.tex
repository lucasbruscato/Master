% Arquivo LaTeX de exemplo de dissertação/tese a ser apresentados à CPG do IME-USP
% 
% Versão 5: Sex Mar  9 18:05:40 BRT 2012
%
% Criação: Jesús P. Mena-Chalco
% Revisão: Fabio Kon e Paulo Feofiloff
%  
% Obs: Leia previamente o texto do arquivo README.txt

\documentclass[12pt,twoside,a4paper]{book}

% ------------------------------------------------------------------- %
% pacotes 
\usepackage{url}
\usepackage[T1]{fontenc} 
\usepackage[portuguese]{babel}
\usepackage[utf8]{inputenc}
\usepackage{float}
\usepackage{amsmath}
\usepackage{amsthm}
\usepackage{amsfonts}
\usepackage{algpseudocode}
\usepackage{amssymb}
\usepackage[pdftex]{graphicx}           % usamos arquivos pdf/png como figuras
\usepackage{setspace}                   % espaçamento flexível
\usepackage{indentfirst}                % indentação do primeiro parágrafo
\usepackage{makeidx}                    % índice remissivo
\usepackage[nottoc]{tocbibind}          % acrescentamos a bibliografia/indice/conteudo no Table of Contents
\usepackage{courier}                    % usa o Adobe Courier no lugar de Computer Modern Typewriter
\usepackage{type1cm}                    % fontes realmente escaláveis
\usepackage{listings}                   % para formatar código-fonte (ex. em Java)
\usepackage{titletoc}
\usepackage[fixlanguage]{babelbib}
\usepackage[font=small,format=plain,labelfont=bf,up,textfont=it,up]{caption}
\usepackage[usenames,svgnames,dvipsnames]{xcolor}
\usepackage[a4paper,top=3.0cm,bottom=2.0cm,left=3.0cm,right=2.0cm]{geometry} % margens
\usepackage[pdftex,plainpages=false,pdfpagelabels,pagebackref,colorlinks=true,citecolor=DarkGreen,linkcolor=NavyBlue,urlcolor=DarkRed,filecolor=green,bookmarksopen=true]{hyperref} % links coloridos
\usepackage[all]{hypcap} % soluciona o problema com o hyperref e capitulos
\usepackage[round,sort,nonamebreak]{natbib} % citação bibliográfica textual(plainnat-ime.bst)
\fontsize{60}{62}\usefont{OT1}{cmr}{m}{n}{\selectfont}

% ------------------------------------------------------------------- %
% Cabeçalhos similares ao TAOCP de Donald E. Knuth
\usepackage{fancyhdr}
\pagestyle{fancy}
\fancyhf{}
\renewcommand{\chaptermark}[1]{\markboth{\MakeUppercase{#1}}{}}
\renewcommand{\sectionmark}[1]{\markright{\MakeUppercase{#1}}{}}
\renewcommand{\headrulewidth}{0pt}

\allowdisplaybreaks

% ------------------------------------------------------------------- %
\graphicspath{{./figuras/}}             % caminho das figuras (recomendável)
\frenchspacing                          % arruma o espaço: id est (i.e.) e exempli gratia (e.g.) 
\urlstyle{same}                         % URL com o mesmo estilo do texto e não mono-spaced
\makeindex                              % para o índice remissivo
\raggedbottom                           % para não permitir espaços extra no texto
\fontsize{60}{62}\usefont{OT1}{cmr}{m}{n}{\selectfont}
\cleardoublepage
\normalsize

% ------------------------------------------------------------------- %
% Opções de listing usados para o código fonte
% Ref: http://en.wikibooks.org/wiki/LaTeX/Packages/Listings
\lstset{
    basicstyle=\ttfamily,           % the size of the fonts that are used for the code
    stepnumber=1,                   % the step between two line-numbers. If it's 1 each line will be numbered
    showspaces=false,               % show spaces adding particular underscores
    showstringspaces=false,         % underline spaces within strings
    showtabs=false,                 % show tabs within strings adding particular underscores
    framerule=0.6pt,
    tabsize=2,                      % sets default tabsize to 2 spaces
    captionpos=b,                   % sets the caption-position to bottom
    breaklines=true,                % sets automatic line breaking
    breakatwhitespace=false,        % sets if automatic breaks should only happen at whitespace
    frame=single,                   % adds a frame around the code
    rulecolor=\color{gray},         % if not set, the frame-color may be changed on line-breaks within not-black text (e.g. comments (green here))
    extendedchars=true,
    xleftmargin=10pt,
    xrightmargin=10pt,
    framexleftmargin=10pt,
    framexrightmargin=10pt
}

% Comandos customizados
\newcommand{\diag}{\operatorname{diag}}
\newcommand{\yi}{{\boldsymbol{y}_{(i)}}}
\newcommand{\bhat}{{\hat{\boldsymbol{\beta}}}}
\newcommand{\sigsq}{\sigma^2}
\newcommand{\tausq}{{\tau^2}}
\newcommand{\ubf}{{\boldsymbol{u}}}
\newcommand{\vbf}{{\boldsymbol{v}}}
\newcommand{\tbf}{{\boldsymbol{t}}}
\newcommand{\ybf}{{\boldsymbol{y}}}
\newcommand{\transp}{{\mathsf{T}}}
\newcommand{\betabf}{{\boldsymbol{\beta}}}
\newtheorem{thm}{Teorema}
\newtheorem{prop}{Proposição}[chapter]
% ------------------------------------------------------------------- %
% Corpo do texto
\begin{document}
\frontmatter 
% cabeçalho para as páginas das seções anteriores ao capítulo 1 (frontmatter)
\fancyhead[RO]{{\footnotesize\rightmark}\hspace{2em}\thepage}
\setcounter{tocdepth}{2}
\fancyhead[LE]{\thepage\hspace{2em}\footnotesize{\leftmark}}
\fancyhead[RE,LO]{}
\fancyhead[RO]{{\footnotesize\rightmark}\hspace{2em}\thepage}
\onehalfspacing  % espaçamento

% ------------------------------------------------------------------- %
% CAPA
% Nota: O título para as dissertações/teses do IME-USP devem caber em um 
% orifício de 10,7cm de largura x 6,0cm de altura que há na capa fornecida pela SPG.
\thispagestyle{empty}
\begin{center}
    \vspace*{2.3cm}
    
    \Large{\textbf{Predição de polaridade negativa\\
    em relatórios de auditoria\\
    utilizando dados socioeconômicos}}
    
    \vspace*{1.2cm}
    \Large{Lucas Peinado Bruscato}
    
    \vskip 2cm
    \textsc{
    Dissertação apresentada\\[-0.25cm] 
    ao\\[-0.25cm]
    Instituto de Matemática e Estatística\\[-0.25cm]
    da\\[-0.25cm]
    Universidade de São Paulo\\[-0.25cm]
    para\\[-0.25cm]
    obtenção do título\\[-0.25cm]
    de\\[-0.25cm]
    Mestre em Ciências}
    
    \vskip 1.5cm
    Programa: Estatística\\
    Orientador: Profa. Dra. Florencia Leonardi

    \vskip 1cm
    \normalsize{}
    
    \vskip 0.5cm
    \normalsize{São Paulo, fevereiro de 2020}
\end{center}


\newpage
\thispagestyle{empty}
\begin{center}
    \vspace*{2.3 cm}
    
    \textbf{\Large{Predição de polaridade negativa\\
    em relatórios de auditoria\\
    utilizando dados socioeconômicos}}\\
    
    \vspace*{2 cm}
\end{center}

\vskip 2cm

\begin{flushright}
    Esta é a versão original da dissertação/tese elaborada pelo\\
    candidato Lucas Peinado Bruscato, tal como\\
    submetida à Comissão Julgadora.
    
    \vskip 2cm

\end{flushright}

\vskip 4.2cm

% ------------------------------------------------------------------- %
% Agradecimentos:
% Se o candidato não quer fazer agradecimentos, deve simplesmente eliminar esta página 
\chapter*{Agradecimentos}

(?).

\pagenumbering{roman}

% ------------------------------------------------------------------- %
% Resumo
\chapter*{Resumo}

(?).

% ------------------------------------------------------------------- %
% Abstract
\chapter*{Abstract}

(?).

% ------------------------------------------------------------------- %
% Sumário
\tableofcontents    % imprime o sumário

% ------------------------------------------------------------------- %
% Listas de figuras e tabelas criadas automaticamente
\listoffigures            
\listoftables            

% ------------------------------------------------------------------- %
% Capítulos do trabalho
\mainmatter

% cabeçalho para as páginas de todos os capítulos
\fancyhead[RE,LO]{\thesection}

\singlespacing              % espaçamento simples
%\onehalfspacing            % espaçamento um e meio

\input c1-introduction % associado ao arquivo: 'c1-introduction.tex'
\input c2-context % associado ao arquivo: 'c2-context.tex'
\input c3-modeling % associado ao arquivo: 'c3-modeling.tex'
\input c4-other_approaches % associado ao arquivo: 'c4-other_approaches.tex'
\input c5-conclusion % associado ao arquivo: 'c5-conclusion.tex'

% cabeçalho para os apêndices
\renewcommand{\chaptermark}[1]{\markboth{\MakeUppercase{\appendixname\ \thechapter}} {\MakeUppercase{#1}} }
\fancyhead[RE,LO]{}
\appendix

\chapter{Códigos em \tt{Python}}
\label{ape:codigos}

\hypertarget{data-processing-raw-dataset}{%
\section{Data Processing (raw
dataset)}\label{data-processing-raw-dataset}}

\begin{lstlisting}[language=Python]
import pandas as pd
import os
\end{lstlisting}

\hypertarget{read-and-handle-target-feature}{%
\subsubsection{Read and handle target
feature}\label{read-and-handle-target-feature}}

\begin{lstlisting}[language=Python]
target_feature = pd.read_csv('../target_feature/01_target_feature.csv',
                             sep=';')
\end{lstlisting}

\begin{lstlisting}[language=Python]
target_feature.head()
\end{lstlisting}

\begin{lstlisting}[language=Python]
target_feature['temp'] = target_feature['file_name'].str.replace('[0-9]|.pdf|-', ' ', regex=True)\
    .str.normalize('NFKD').str.encode('ascii', errors='ignore').str.decode('utf-8').str.lower().str.strip()
target_feature['city'] = target_feature['temp'].str[:-3]
target_feature['state'] = target_feature['temp'].str[-2:]
target_feature['city_state'] = target_feature['city'].map(str) + '_' + target_feature['state']

target_feature = target_feature.drop("temp", axis=1)
target_feature.head()
\end{lstlisting}

\hypertarget{read-explanatory-features-education-family-fertility-and-work-2000-and-2010}{%
\subsubsection{Read explanatory features: education, family, fertility
and work (2000 and
2010)}\label{read-explanatory-features-education-family-fertility-and-work-2000-and-2010}}

\begin{lstlisting}[language=Python]
raw_dataset = target_feature
\end{lstlisting}

\begin{lstlisting}[language=Python]
state_name_to_acronym = pd.DataFrame({'full_state_name': 
                                      ['acre', 
                                       'alagoas', 
                                       'amapa', 
                                       'amazonas', 
                                       'bahia', 
                                       'ceara', 
                                       'distrito_federal', 
                                       'espirito_santo', 
                                       'goias', 
                                       'maranhao', 
                                       'mato_grosso', 
                                       'mato_grosso_do_sul', 
                                       'minas_gerais', 
                                       'para', 
                                       'paraiba', 
                                       'parana', 
                                       'pernambuco', 
                                       'piaui', 
                                       'rio_de_janeiro', 
                                       'rio_grande_do_norte', 
                                       'rio_grande_do_sul', 
                                       'rondonia', 
                                       'roraima', 
                                       'santa_catarina', 
                                       'sao_paulo', 
                                       'sergipe', 
                                       'tocantins'],
                                      'acronym': ['ac',
                                                  'al',
                                                  'ap',
                                                  'am',
                                                  'ba',
                                                  'ce',
                                                  'df',
                                                  'es',
                                                  'go',
                                                  'ma',
                                                  'mt',
                                                  'ms',
                                                  'mg',
                                                  'pa',
                                                  'pb',
                                                  'pr',
                                                  'pe',
                                                  'pi',
                                                  'rj',
                                                  'rn',
                                                  'rs',
                                                  'ro',
                                                  'rr',
                                                  'sc',
                                                  'sp',
                                                  'se',
                                                  'to']})

var_list = ['var_01',
            'var_02',
            'var_03']
\end{lstlisting}

\begin{lstlisting}[language=Python]
paths = ['../ibge_censo/2000/education',
         '../ibge_censo/2000/family',
         '../ibge_censo/2000/fertility',
         '../ibge_censo/2000/work',
         '../ibge_censo/2010/education',
         '../ibge_censo/2010/family',
         '../ibge_censo/2010/fertility',
         '../ibge_censo/2010/work']
\end{lstlisting}

\begin{lstlisting}[language=Python]
for path in paths:
    
    for var_name in var_list:
        full_temp = pd.DataFrame()
        
        for state in os.listdir(path):
            if not state.startswith('.'):
                state_acronym = state_name_to_acronym.loc[
                    state_name_to_acronym.full_state_name == state]['acronym'].values[0]
        
                for filename in os.listdir(path + '/' + state):
                    if not filename.startswith('.') and filename.endswith(var_name + '.csv'):
        
                        temp = pd.read_csv(path + '/' + state + '/' + filename)
                        temp['city_state'] = temp['city'].map(str) + '_' + state_acronym
                        
                        full_temp = pd.concat([full_temp, temp])
        
        if full_temp.shape[0] != 0:
            full_temp = full_temp.add_prefix(path.split("/")[2] + '_' + path.split("/")[3] + '_' + var_name + '_')
            column_to_join = path.split("/")[2] + '_' + path.split("/")[3] + '_' + var_name + '_city_state'
            
            raw_dataset = pd.merge(raw_dataset,
                                   full_temp.iloc[:,1:],
                                   left_on="city_state",
                                   right_on=column_to_join,
                                   how="left")
            
            raw_dataset = raw_dataset.drop(column_to_join, axis=1)
            
            print(path + ' [' + var_name + '] ')
        
\end{lstlisting}

\hypertarget{read-explanatory-feature-social-indicator-different-pattern}{%
\subsubsection{Read explanatory feature: social indicator (different
pattern)}\label{read-explanatory-feature-social-indicator-different-pattern}}

\begin{lstlisting}[language=Python]
paths = ['../ibge_censo/2010/social_indicator']
\end{lstlisting}

\hypertarget{changing-city-name-due-to-city-being-known-by-two-different-names}{%
\subsubsection{Changing city name due to city being known by two
different
names}\label{changing-city-name-due-to-city-being-known-by-two-different-names}}

\begin{lstlisting}[language=Python]
raw_dataset.loc[raw_dataset.file_name=='3238-São Valério da Natividade-TO.pdf', 'city_state'] = 'sao valerio_to'
\end{lstlisting}

\begin{lstlisting}[language=Python]
for path in paths:
    
    for var_name in var_list:
        full_temp = pd.DataFrame()
        
        for state in os.listdir(path):
            if not state.startswith('.'):
                state_acronym = state_name_to_acronym.loc[
                    state_name_to_acronym.full_state_name == state]['acronym'].values[0]
                
                for filename in os.listdir(path + '/' + state):
                    if not filename.startswith('.') and filename.endswith(var_name + '.csv'):
                        
                        temp = pd.read_csv(path + '/' + state + '/' + filename)
                        temp['city_state'] = temp['city'].map(str) + '_' + state_acronym
                        
                        full_temp = pd.concat([full_temp, temp])
                        
        if full_temp.shape[0] != 0:
            full_temp = full_temp.add_prefix(path.split("/")[3] + '_' + var_name + '_')
            column_to_join = path.split("/")[3] + '_' + var_name + '_city_state'
            
            raw_dataset = pd.merge(raw_dataset,
                                   full_temp.iloc[:,1:],
                                   left_on="city_state",
                                   right_on=column_to_join,
                                   how="left")
            
            raw_dataset = raw_dataset.drop(column_to_join, axis=1)
            
            print(path + ' [' + var_name + '] ')
\end{lstlisting}

\hypertarget{read-explanatory-feature-enem-score-different-pattern}{%
\subsubsection{Read explanatory feature: enem score (different
pattern)}\label{read-explanatory-feature-enem-score-different-pattern}}

\begin{lstlisting}[language=Python]
paths = ['../enem/2000/2000_enem_score_var_01.csv',
         '../enem/2010/2010_enem_score_var_01.csv']
\end{lstlisting}

\begin{lstlisting}[language=Python]
for path in paths:
    temp = pd.read_csv(path, sep=';')
    temp = temp.iloc[1:,:]
    temp = temp.add_prefix(path.split("/")[2] + '_enem_var_01_')
    column_to_join = path.split("/")[2] + '_enem_var_01_city_state'

    raw_dataset = pd.merge(raw_dataset,
                           temp,
                           left_on="city_state",
                           right_on=column_to_join,
                           how="left")
    
    raw_dataset = raw_dataset.drop(column_to_join, axis=1)
    
raw_dataset.iloc[:, 88:95] = raw_dataset.iloc[:, 88:95].fillna(0)
\end{lstlisting}

\begin{lstlisting}[language=Python]
for column in raw_dataset.columns:
    print(column)
\end{lstlisting}

\begin{lstlisting}[language=Python]
raw_dataset.to_csv('02_01_raw_dataset.csv',
                   sep=';',
                   index=False)
\end{lstlisting}



\label{ape:data_processing_modeling_dataset}
\hypertarget{data-processing-modeling-dataset}{%
\section{Data Processing (modeling
dataset)}\label{data-processing-modeling-dataset}}

\begin{lstlisting}[language=Python]
import pandas as pd
import numpy as np
\end{lstlisting}

\hypertarget{feature-engineering}{%
\subsubsection{Feature engineering}\label{feature-engineering}}

\begin{lstlisting}[language=Python]
full_dataset = pd.read_csv('02_01_raw_dataset.csv',
                           sep=';')
\end{lstlisting}

\begin{lstlisting}[language=Python]
full_dataset.head()
\end{lstlisting}

\begin{lstlisting}[language=Python]
full_dataset = full_dataset.replace('-', 0)

full_dataset.iloc[:, 10:] = full_dataset.iloc[:, 10:].apply(pd.to_numeric)
\end{lstlisting}

For all the features created using IBGE, divide them from the position
in 2000 by the position in 2010, with the following observations:
\textbf{if the feature is not a proportion, then divide the feature by
the population size of the year (2000 or 2010)}

\begin{lstlisting}[language=Python]
full_dataset['education_var_01_quantity_pct'] = (full_dataset['2000_education_var_01_quantity'] / full_dataset['2000_family_var_02_quantity']) / (full_dataset['2010_education_var_01_quantity'] / full_dataset['2010_family_var_02_quantity'])

full_dataset['family_var_01_suitable_pct'] = (full_dataset['2000_family_var_01_suitable'] / full_dataset['2000_family_var_01_total']) / (full_dataset['2010_family_var_01_suitable'] / full_dataset['2010_family_var_01_total'])
full_dataset['family_var_01_semi_suitable_pct'] = (full_dataset['2000_family_var_01_semi_suitable'] / full_dataset['2000_family_var_01_total']) / (full_dataset['2010_family_var_01_semi_suitable'] / full_dataset['2010_family_var_01_total'])
full_dataset['family_var_01_inappropriate_pct'] = (full_dataset['2000_family_var_01_inappropriate'] / full_dataset['2000_family_var_01_total']) / (full_dataset['2010_family_var_01_inappropriate'] / full_dataset['2010_family_var_01_total'])

full_dataset['fertility_var_01_has_children_pct'] = (full_dataset['2000_fertility_var_01_has_children'] / full_dataset['2000_fertility_var_01_total']) / (full_dataset['2010_fertility_var_01_has_children'] / full_dataset['2010_fertility_var_01_total'])
full_dataset['fertility_var_01_children_born_pct'] = (full_dataset['2000_fertility_var_01_children_born'] / full_dataset['2000_fertility_var_01_total']) / (full_dataset['2010_fertility_var_01_children_born'] / full_dataset['2010_fertility_var_01_total'])
full_dataset['fertility_var_01_children_borned_live_pct'] = (full_dataset['2000_fertility_var_01_children_borned_live'] / full_dataset['2000_fertility_var_01_total']) / (full_dataset['2010_fertility_var_01_children_borned_live'] / full_dataset['2010_fertility_var_01_total'])
full_dataset['fertility_var_01_children_borned_dead_pct'] = (full_dataset['2000_fertility_var_01_children_borned_dead'] / full_dataset['2000_fertility_var_01_total']) / (full_dataset['2010_fertility_var_01_children_borned_dead'] / full_dataset['2010_fertility_var_01_total'])

full_dataset['fertility_var_02_married_pct'] = (full_dataset['2000_fertility_var_02_married'] / full_dataset['2000_fertility_var_02_total']) / (full_dataset['2010_fertility_var_02_married'] / full_dataset['2010_fertility_var_02_total'])
full_dataset['fertility_var_02_separated_pct'] = (full_dataset['2000_fertility_var_02_separated'] / full_dataset['2000_fertility_var_02_total']) / (full_dataset['2010_fertility_var_02_separated'] / full_dataset['2010_fertility_var_02_total'])
full_dataset['fertility_var_02_divorced_pct'] = (full_dataset['2000_fertility_var_02_divorced'] / full_dataset['2000_fertility_var_02_total']) / (full_dataset['2010_fertility_var_02_divorced'] / full_dataset['2010_fertility_var_02_total'])
full_dataset['fertility_var_02_widow_pct'] = (full_dataset['2000_fertility_var_02_widow'] / full_dataset['2000_fertility_var_02_total']) / (full_dataset['2010_fertility_var_02_widow'] / full_dataset['2010_fertility_var_02_total'])
full_dataset['fertility_var_02_single_pct'] = (full_dataset['2000_fertility_var_02_single'] / full_dataset['2000_fertility_var_02_total']) / (full_dataset['2010_fertility_var_02_single'] / full_dataset['2010_fertility_var_02_total'])

full_dataset['fertility_var_03_total_pct'] = (full_dataset['2000_fertility_var_03_total'] / full_dataset['2000_family_var_02_quantity']) / (full_dataset['2010_fertility_var_03_total'] / full_dataset['2010_family_var_02_quantity'])

full_dataset['work_var_01_regular_pct'] = ((full_dataset['2000_work_var_01_domestic_regular'] + full_dataset['2000_work_var_01_other_regular'] + full_dataset['2000_work_var_01_military_and_gov']) / full_dataset['2000_work_var_01_total']) / ((full_dataset['2010_work_var_01_main_regular'] + full_dataset['2010_work_var_01_other_regular']) / full_dataset['2010_work_var_01_total'])
full_dataset['work_var_01_irregular_pct'] = ((full_dataset['2000_work_var_01_domestic_irregular'] + full_dataset['2000_work_var_01_other_irregular']) / full_dataset['2000_work_var_01_total']) / ((full_dataset['2010_work_var_01_main_irregular'] + full_dataset['2010_work_var_01_other_irregular']) / full_dataset['2010_work_var_01_total'])

full_dataset['social_indicator_var_01_15_to_24_years_pct'] = (full_dataset['social_indicator_var_01_2000_15_to_24_years'] / full_dataset['social_indicator_var_01_2010_15_to_24_years'])
full_dataset['social_indicator_var_01_25_to_59_years_pct'] = (full_dataset['social_indicator_var_01_2000_25_to_59_years'] / full_dataset['social_indicator_var_01_2010_25_to_59_years'])
full_dataset['social_indicator_var_01_60_to_more_years_pct'] = (full_dataset['social_indicator_var_01_2000_60_to_more_years'] / full_dataset['social_indicator_var_01_2010_60_to_more_years'])

full_dataset['social_indicator_var_02_suitable_pct'] = full_dataset['social_indicator_var_02_2000_suitable'] / full_dataset['social_indicator_var_02_2010_suitable']
full_dataset['social_indicator_var_02_semi_suitable_pct'] = full_dataset['social_indicator_var_02_2000_semi_suitable'] / full_dataset['social_indicator_var_02_2010_semi_suitable']
full_dataset['social_indicator_var_02_inappropriate_pct'] = full_dataset['social_indicator_var_02_2000_inappropriate'] / full_dataset['social_indicator_var_02_2010_inappropriate']

full_dataset['social_indicator_var_03_responsable_illiterate_pct'] = full_dataset['social_indicator_var_03_2000_responsable_illiterate'] / full_dataset['social_indicator_var_03_2010_responsable_illiterate']
full_dataset['social_indicator_var_03_inappropriate_residence_pct'] = full_dataset['social_indicator_var_03_2000_inappropriate_residence'] / full_dataset['social_indicator_var_03_2010_inappropriate_residence']
full_dataset['social_indicator_var_03_responsable_illiterate_and_inapprop riate_residence_pct'] = full_dataset['social_indicator_var_03_2000_responsable_illiterate_and_ina ppropriate_residence'] / full_dataset['social_indicator_var_03_2010_responsable_illiterate_and_ina
ppropriate_residence']

full_dataset['enem_var_01_enem_score_mean_pct'] = full_dataset['2000_enem_var_01_enem_score_mean'] / full_dataset['2010_enem_var_01_enem_score_mean']
full_dataset['enem_var_01_enem_score_std_pct'] = full_dataset['2000_enem_var_01_enem_score_std'] / full_dataset['2010_enem_var_01_enem_score_std']
full_dataset['enem_var_01_enem_score_median_pct'] = full_dataset['2000_enem_var_01_enem_score_median'] / full_dataset['2010_enem_var_01_enem_score_median']
\end{lstlisting}

For the target feature, change the proportion getting the relative
\textbf{pct\_pol\_neg} over the sum of \textbf{pct\_pol\_neg} and
\textbf{pct\_pol\_pos}

\begin{lstlisting}[language=Python]
full_dataset.insert(0, 'pct_pol_neg_rel', full_dataset['pct_pol_neg']/(full_dataset['pct_pol_neg']+full_dataset['pct_pol_pos']))
\end{lstlisting}

\hypertarget{select-only-generated-features-and-remove-rows-with-nan-values-on-it}{%
\subsubsection{Select only generated features and remove rows with NaN
values on
it}\label{select-only-generated-features-and-remove-rows-with-nan-values-on-it}}

\begin{lstlisting}[language=Python]
modeling_dataset = full_dataset.iloc[:, np.r_[0, 9, 96:124]].dropna()
\end{lstlisting}

\hypertarget{remove-inf-value-for-all-columns}{%
\subsubsection{Remove `inf' value for all
columns}\label{remove-inf-value-for-all-columns}}

\begin{lstlisting}[language=Python]
def replace_inf_by_max(df, col_name):
    max_for_column = max(df.loc[df[col_name] != np.inf, col_name])
    df.loc[df[col_name] == np.inf, col_name] = max_for_column
    
    return df
\end{lstlisting}

\begin{lstlisting}[language=Python]
for col in modeling_dataset.columns:
    if col not in ['enem_var_01_enem_score_mean_pct', 'enem_var_01_enem_score_std_pct', 'enem_var_01_enem_score_median_pct']:
        modeling_dataset = replace_inf_by_max(modeling_dataset, col)
\end{lstlisting}

\hypertarget{one-hot-encoding-for-state-feature}{%
\subsubsection{One hot encoding for `state'
feature}\label{one-hot-encoding-for-state-feature}}

\begin{lstlisting}[language=Python]
modeling_dataset = pd.concat([
    modeling_dataset,
    pd.get_dummies(modeling_dataset['state'], prefix = 'state')
], axis = 1)

modeling_dataset = modeling_dataset.drop(columns=['state'])
\end{lstlisting}

\begin{lstlisting}[language=Python]
modeling_dataset.tail()
\end{lstlisting}

\begin{lstlisting}[language=Python]
for col in modeling_dataset.columns:
    print(col)
\end{lstlisting}

\begin{lstlisting}[language=Python]
modeling_dataset.to_csv('02_02_modeling_dataset.csv',
                        sep=';',
                        index=False)
\end{lstlisting}


\section{Data Processing}\label{data-processing}

\begin{lstlisting}[language=Python]
import pandas as pd
\end{lstlisting}

\subsubsection{Splitting the datasets (training and
validation)}\label{splitting-the-datasets-training-and-validation}

\begin{lstlisting}[language=Python]
modeling_dataset = pd.read_csv('02_02_modeling_dataset.csv',
                               sep=';')
\end{lstlisting}

\begin{lstlisting}[language=Python]
training_dataset = modeling_dataset.sample(frac=0.75,
                                           random_state=7)
validation_dataset = modeling_dataset.drop(training_dataset.index)
\end{lstlisting}

\begin{lstlisting}[language=Python]
print('=== Number of rows === \n' +
      'Training: ' + str(len(training_dataset)) + '\n' +
      'Validation: ' + str(len(validation_dataset)))
\end{lstlisting}

\begin{lstlisting}
=== Number of rows === 
Training: 422
Validation: 141
\end{lstlisting}

\begin{lstlisting}[language=Python]
training_dataset.head()
\end{lstlisting}

\begin{lstlisting}
<tr style="text-align: right;">
  <th></th>
  <th>pct_pol_neg</th>
  <th>pct_pol_pos</th>
  <th>pct_pol_neu</th>
  <th>education_var_01_qt_pct</th>
  <th>family_var_01_adequada_pct</th>
  <th>family_var_01_semi_adequada_pct</th>
  <th>family_var_01_inadequada_pct</th>
  <th>fertility_var_01_has_children_pct</th>
  <th>fertility_var_01_children_born_pct</th>
  <th>fertility_var_01_children_borned_live_pct</th>
  <th>...</th>
  <th>state_pr</th>
  <th>state_rj</th>
  <th>state_rn</th>
  <th>state_ro</th>
  <th>state_rr</th>
  <th>state_rs</th>
  <th>state_sc</th>
  <th>state_se</th>
  <th>state_sp</th>
  <th>state_to</th>
</tr>
\end{lstlisting}

\begin{lstlisting}
<tr>
  <th>408</th>
  <td>0.017929</td>
  <td>0.026417</td>
  <td>0.958357</td>
  <td>1.021423</td>
  <td>0.054541</td>
  <td>1.033242</td>
  <td>9.985730</td>
  <td>0.989622</td>
  <td>1.090811</td>
  <td>1.082467</td>
  <td>...</td>
  <td>0</td>
  <td>0</td>
  <td>0</td>
  <td>0</td>
  <td>0</td>
  <td>1</td>
  <td>0</td>
  <td>0</td>
  <td>0</td>
  <td>0</td>
</tr>
<tr>
  <th>188</th>
  <td>0.013723</td>
  <td>0.025105</td>
  <td>0.964276</td>
  <td>1.058820</td>
  <td>1.178352</td>
  <td>0.638225</td>
  <td>24.733100</td>
  <td>1.010927</td>
  <td>1.133640</td>
  <td>1.122890</td>
  <td>...</td>
  <td>0</td>
  <td>0</td>
  <td>0</td>
  <td>0</td>
  <td>0</td>
  <td>0</td>
  <td>1</td>
  <td>0</td>
  <td>0</td>
  <td>0</td>
</tr>
<tr>
  <th>97</th>
  <td>0.012676</td>
  <td>0.025575</td>
  <td>0.963083</td>
  <td>0.991957</td>
  <td>0.641255</td>
  <td>1.170128</td>
  <td>9.192867</td>
  <td>0.999439</td>
  <td>1.147375</td>
  <td>1.140280</td>
  <td>...</td>
  <td>0</td>
  <td>0</td>
  <td>0</td>
  <td>0</td>
  <td>0</td>
  <td>1</td>
  <td>0</td>
  <td>0</td>
  <td>0</td>
  <td>0</td>
</tr>
<tr>
  <th>431</th>
  <td>0.021631</td>
  <td>0.030575</td>
  <td>0.949254</td>
  <td>0.970031</td>
  <td>0.876919</td>
  <td>1.228749</td>
  <td>2.848914</td>
  <td>0.962917</td>
  <td>1.129891</td>
  <td>1.123553</td>
  <td>...</td>
  <td>0</td>
  <td>0</td>
  <td>0</td>
  <td>0</td>
  <td>0</td>
  <td>0</td>
  <td>0</td>
  <td>0</td>
  <td>0</td>
  <td>0</td>
</tr>
<tr>
  <th>475</th>
  <td>0.014607</td>
  <td>0.031221</td>
  <td>0.955085</td>
  <td>0.974727</td>
  <td>2.738600</td>
  <td>0.943244</td>
  <td>2.845043</td>
  <td>0.977073</td>
  <td>1.233269</td>
  <td>1.185966</td>
  <td>...</td>
  <td>1</td>
  <td>0</td>
  <td>0</td>
  <td>0</td>
  <td>0</td>
  <td>0</td>
  <td>0</td>
  <td>0</td>
  <td>0</td>
  <td>0</td>
</tr>
\end{lstlisting}

5 rows × 52 columns

\begin{lstlisting}[language=Python]
training_dataset.to_csv('02_03_training_dataset.csv',
                        sep=';',
                        index=False)
\end{lstlisting}

\begin{lstlisting}[language=Python]
validation_dataset.to_csv('02_03_validation_dataset.csv',
                          sep=';',
                          index=False)
\end{lstlisting}


\hypertarget{create-target-feature}{%
\section{Create Target Feature}\label{create-target-feature}}

\begin{lstlisting}[language=Python]
import PyPDF2
import unidecode
import pandas as pd
from collections import Counter
import csv
import os
import os.path
import random
import re
import datetime
\end{lstlisting}

\hypertarget{create-lists-with-all-paths-to-all-files}{%
\subsubsection{Create lists with all paths to all
files}\label{create-lists-with-all-paths-to-all-files}}

\begin{lstlisting}[language=Python]
folders = ["ciclo_3",
           "ciclo_4",
           "ciclo_5",
           "edicoes_anteriores/sorteio_34",
           "edicoes_anteriores/sorteio_35",
           "edicoes_anteriores/sorteio_36",
           "edicoes_anteriores/sorteio_37",
           "edicoes_anteriores/sorteio_38",
           "edicoes_anteriores/sorteio_39",
           "edicoes_anteriores/sorteio_40"]

seq_folders = []
file_names = []
file_names_and_paths = []

for folder in folders:
    directory = '../programa_de_fiscalizacao_em_entes_federativos/' + folder
    
    number_of_files = len([name for name in os.listdir(directory) if os.path.isfile(os.path.join(directory, name))])
    
    for i in range(0, number_of_files):
        file_name_and_path = directory + "/" + os.listdir(directory)[i]
        if (".pdf" in file_name_and_path):
            seq_folders.append(folder)
            file_names.append(os.listdir(directory)[i])
            file_names_and_paths.append(file_name_and_path)

print('Example: \n' + file_names_and_paths[0:1][0])
\end{lstlisting}

\hypertarget{generate-target-feature-for-each-report-read-and-summarised-the-polarity}{%
\subsubsection{Generate target feature for each report (`read' and
summarised the
polarity)}\label{generate-target-feature-for-each-report-read-and-summarised-the-polarity}}

\begin{lstlisting}[language=Python]
sentilex_database = pd.read_csv("../sentilex/99_01_sentilex_database.csv",
                                sep = ";")

sentilex_database.adjective = sentilex_database.adjective.str.normalize('NFKD').\
                                str.encode('ascii', errors='ignore').str.decode('utf-8')
\end{lstlisting}

\begin{lstlisting}[language=Python]
cities = pd.DataFrame()

print("List of reports read and summarised")

for file_number in range(0, len(file_names_and_paths)):
    folder = seq_folders[file_number]
    file_name = file_names[file_number]
    file_name_and_path = file_names_and_paths[file_number]
    print(str(datetime.datetime.now()) + ' ' + file_name_and_path)
    
    # read report using external library pdf miner and save in 'temp_report.txt'
    command_to_cmd = 'pdf2txt.py "' + file_name_and_path + '" > temp_report.txt'
    os.system(command_to_cmd)
    
    # read temporary file
    temporary_file = open('temp_report.txt', 'r')
    
    whole_text = ''
    
    for line in temporary_file:
        whole_text += line
    
    words = re.findall(r"[\w']+", unidecode.unidecode(re.sub('\d', ' ', whole_text).lower()))

    # create the frequencies
    words_freq = pd.DataFrame.from_dict(Counter(words), orient = 'index').reset_index()
    words_freq.columns = ['word', 'freq']
    words_freq['pct'] = words_freq['freq']/sum(words_freq.freq)

    # aggregate polarity
    words_freq_polarity = words_freq.merge(sentilex_database,
                                           left_on = "word",
                                           right_on = "adjective",
                                           how = "left").iloc[:, [0, 1, 2, 4]]
    
    # summarise
    number_of_words = words_freq_polarity.freq.sum()
    pct_pol_neg = words_freq_polarity[words_freq_polarity.polarity == -1].pct.sum()
    pct_pol_pos = words_freq_polarity[words_freq_polarity.polarity == 1].pct.sum()
    pct_pol_neu = words_freq_polarity[words_freq_polarity.polarity == 0].pct.sum()
    pct_pol_missing = words_freq_polarity[words_freq_polarity.polarity.isna()].pct.sum()

    current_city = pd.DataFrame({"folder": folder,
                                 "file_name": file_name,
                                 "number_of_words": number_of_words,
                                 "pct_pol_neg": pct_pol_neg,
                                 "pct_pol_pos": pct_pol_pos,
                                 "pct_pol_neu": pct_pol_neu,
                                 "pct_pol_missing": pct_pol_missing},
                                index = [0])
    
    cities = cities.append(current_city)
    
    # save last words_freq_polarity dataframe as an example
    if file_number + 1 == len(file_names_and_paths):
        words_freq_polarity.to_csv('temp_words_freq_polarity.csv',
                                   sep=';',
                                   encoding='utf-8',
                                   index=False)
\end{lstlisting}

\begin{lstlisting}[language=Python]
len(file_names_and_paths)
\end{lstlisting}

\begin{lstlisting}[language=Python]
os.remove("temp_report.txt")
\end{lstlisting}

\begin{lstlisting}[language=Python]
cities.to_csv("../target_feature/01_target_feature.csv",
              sep=';',
              encoding='utf-8',
              index=False)
\end{lstlisting}


\hypertarget{create-initial-sentilex-database-and-improve-it}{%
\section{Create Initial SentiLex Database and Improve
it}\label{create-initial-sentilex-database-and-improve-it}}

\begin{lstlisting}[language=Python]
import csv
import pandas as pd
import os
import os.path
import random
import PyPDF2
import unidecode
import pandas as pd
from collections import Counter
import csv
\end{lstlisting}

\hypertarget{read-sentilex-pt02-and-extract-polarity}{%
\subsubsection{Read SentiLex-PT02 and extract
polarity}\label{read-sentilex-pt02-and-extract-polarity}}

\begin{lstlisting}[language=Python]
# read csv file
sentilex_database = pd.read_csv("SentiLex-flex-PT02.txt", header = None)
sentilex_database.columns = ["adjective", "description"]

# extract "polarity" from "description"
polarity = pd.DataFrame(sentilex_database.description.str.split('\;+').str[3].str.split('\=+').str[1])
sentilex_database = pd.concat([sentilex_database, polarity], axis = 1, join = 'outer')

# remove duplicates
sentilex_database = sentilex_database.iloc[:, [0, 2]].drop_duplicates()
sentilex_database.columns = ["adjective", "polarity"]

# select only polarities in [-1, 0, 1]
polarities = ["-1", "0", "1"]
sentilex_database = sentilex_database[sentilex_database.polarity.isin(polarities)]
\end{lstlisting}

\begin{lstlisting}[language=Python]
sentilex_database.head()
\end{lstlisting}

\hypertarget{save-initial-sentilex-database}{%
\subsubsection{Save initial sentilex
database}\label{save-initial-sentilex-database}}

\begin{lstlisting}[language=Python]
sentilex_database.to_csv("99_01_sentilex_database.csv",
                         sep = ';',
                         encoding = 'utf-8',
                         index = False)
\end{lstlisting}

\hypertarget{define-randomly-reports-for-improving-sentilex-pt02}{%
\subsubsection{Define randomly reports for improving
SentiLex-PT02}\label{define-randomly-reports-for-improving-sentilex-pt02}}

\begin{lstlisting}[language=Python]
folders = ["ciclo_3",
           "ciclo_4",
           "ciclo_5",
           "edicoes_anteriores/sorteio_34",
           "edicoes_anteriores/sorteio_35",
           "edicoes_anteriores/sorteio_36",
           "edicoes_anteriores/sorteio_37",
           "edicoes_anteriores/sorteio_38",
           "edicoes_anteriores/sorteio_39",
           "edicoes_anteriores/sorteio_40"]

file_names_and_paths = []

for folder in folders:
    directory = '../programa_de_fiscalizacao_em_entes_federativos/' + folder
    
    number_of_files = len([name for name in os.listdir(directory) if os.path.isfile(os.path.join(directory, name))]) - 3
    random.seed(7)
    random_file_number = int(random.uniform(0, number_of_files))
    
    file_name_and_path = directory + "/" + os.listdir(directory)[random_file_number]
    file_names_and_paths.append(file_name_and_path)
    
file_names_and_paths
\end{lstlisting}

\hypertarget{import-reports-collect-unique-words-and-save-words-not-in-sentilex-pt02}{%
\subsubsection{Import reports, collect unique words and save words not
in
SentiLex-PT02}\label{import-reports-collect-unique-words-and-save-words-not-in-sentilex-pt02}}

\begin{lstlisting}[language=Python]
print("List of reports read to improve SentiLex database")

words_without_polarity_full = pd.DataFrame(columns=['adjective', 'polarity'])

for file_number in range(0, len(file_names_and_paths)):
    
    file_name = file_names_and_paths[file_number]
    print(file_name)
    
    # create a pdf object
    file = open(file_name, 'rb')
    
    # create a pdf reader object
    file_reader = PyPDF2.PdfFileReader(file)

    # iterate all documents
    word_index = -1
    flag_in_a_word = 0
    words = []

    for i in range(file_reader.numPages):
        page = unidecode.unidecode(file_reader.getPage(i).extractText().lower())

        for j in range(len(page)):
            letter = page[j]

            if (not letter.isalpha()) and flag_in_a_word != 0:
                flag_in_a_word = 0
            elif letter.isalpha() and flag_in_a_word == 0:
                flag_in_a_word = 1
                word_index += 1
                words.append(letter)
            elif letter.isalpha() and flag_in_a_word != 0:
                words[word_index] += letter

    words_unique = pd.DataFrame(pd.DataFrame(words).iloc[:, 0].unique())
    words_unique.columns = ["adjective"]
    
    words_with_polarity = words_unique.merge(sentilex_database,
                                             left_on="adjective",
                                             right_on="adjective",
                                             how="left")
    
    words_without_polarity_full = pd.concat([words_without_polarity_full,
                                             words_with_polarity[words_with_polarity.polarity.isnull()]])


words_without_polarity_full = pd.DataFrame(words_without_polarity_full.adjective.unique())
words_without_polarity_full.columns = ['adjective']
words_without_polarity_full['polarity'] = ''

words_without_polarity_full.sort_values(by=['adjective'], inplace=True)

words_without_polarity_full.to_csv("improving_sentilex/99_create_improving_sentilex.csv",
                                   sep=';',
                                   encoding='utf-8',
                                   index=False)
\end{lstlisting}

\section{Extract Explanatory
Features}\label{extract-explanatory-features}

\begin{lstlisting}[language=Python]
import os
import shutil
import pandas as pd
\end{lstlisting}

\subsubsection{Script to select needed values for building
features}\label{script-to-select-needed-values-for-building-features}

\begin{lstlisting}[language=Python]
def select_feature_range(df, columns_index, initial_row):
    size = 0
    i = 0
    
    if initial_row != 0:
        initial_row -= 2
        size += 1
    
    for identifier in df.iloc[initial_row:, columns_index[0]]:
        i += 1
        if identifier == 'Municípios':
            initial_row = i
        
        if initial_row != 0:
            size += 1
            if type(identifier) == float or identifier[:6] == 'Fonte:':
                break

    return df.iloc[initial_row:(initial_row+size-2), columns_index]
\end{lstlisting}

\begin{lstlisting}[language=Python]
def create_feature_files(year, subject, feature_identification, columns_index, 
                         name_str_identification_prefix, name_str_identification_suffix,
                         number_of_character_prefix, number_of_character_suffix,
                         col_names, path, states, is_to_copy, initial_row):
    quantity_of_features_created = 0
    
    for state in states:
    
        if not state.startswith('.') and is_to_copy:
            files = os.listdir(path + state)
            
            for file in files:
                if ((name_str_identification_prefix == None or 
                    file[:number_of_character_prefix] == name_str_identification_prefix) and 
                    (file[len(file)-number_of_character_suffix:] == name_str_identification_suffix)):
                    
                    df = pd.read_excel(path + state + '/' + file)

                    df_var = select_feature_range(df, columns_index, initial_row)
                    df_var.columns = col_names

                    # remove accents and lower case
                    df_var.city = df_var.city.str.normalize('NFKD').str.encode('ascii', errors='ignore').\
                        str.decode('utf-8').str.lower()

                    # save in a csv file
                    df_var.to_csv(path + state + '/' + year + '_' + subject + '_' + state + '_var_' +
                                      feature_identification + '.csv',
                                  index=False,
                                  sep = ',')
                    
                    print(path + state + '/' + year + '_' + subject + '_' + state + '_var_' + 
                              feature_identification + '.csv')
                    quantity_of_features_created += 1
                    
    print('Quantity of features created: ' + str(quantity_of_features_created))
\end{lstlisting}

\subsubsection{\texorpdfstring{For security reasons, change parameter
`is\_to\_copy' to True before start
running}{For security reasons, change parameter is\_to\_copy to True before start running}}\label{for-security-reasons-change-parameter-is_to_copy-to-true-before-start-running}

\begin{lstlisting}[language=Python]
# set parameter here #
is_to_copy = False
######################
\end{lstlisting}

\subsubsection{\texorpdfstring{Set initial parameters to select needed
values for building \emph{each}
features}{Set initial parameters to select needed values for building each features}}\label{set-initial-parameters-to-select-needed-values-for-building-each-features}

\begin{lstlisting}[language=Python]
# set parameters here #
year = '2010'
subject = 'family'
feature_identification = '02'
columns_index = [0, 4]
name_str_identification_prefix = 'tab4_' # String or None if don't need a suffix to search
name_str_identification_suffix = '2_1.xls'
col_names = ['city',
             'qt']
initial_row = 0 # integer or 0 if it follows the structure (python pattern)
#######################

number_of_character_suffix = len(name_str_identification_suffix)
path = year + '/' + subject + '/'
states = os.listdir(path)

print(path)
print(feature_identification)

if name_str_identification_prefix != None:
    number_of_character_prefix = len(name_str_identification_prefix)
    print(name_str_identification_prefix + '-*-' + name_str_identification_suffix)
else:
    number_of_character_prefix = 0
    print('-*-' + name_str_identification_suffix)
\end{lstlisting}

\begin{lstlisting}
2010/family/
02
tab4_-*-2_1.xls
\end{lstlisting}

\begin{lstlisting}[language=Python]
create_feature_files(year, subject, feature_identification, columns_index, 
                     name_str_identification_prefix, name_str_identification_suffix,
                     number_of_character_prefix, number_of_character_suffix,
                     col_names, path, states, is_to_copy, initial_row)
\end{lstlisting}

\begin{lstlisting}
Quantity of features created: 0
\end{lstlisting}



\label{ape:import_reports}
\hypertarget{import-reports}{%
\section{Import reports}\label{import-reports}}

\begin{lstlisting}[language=Python]
from selenium import webdriver
from selenium.webdriver.support.ui import Select
from selenium.webdriver.common.keys import Keys
from selenium.webdriver.firefox.firefox_profile import FirefoxProfile

import os
import time
import codecs
\end{lstlisting}

\hypertarget{set-configurations-only-for-edicoes_anteriores-others-are-manually}{%
\subsubsection{Set configurations (only for `edicoes\_anteriores',
others are
manually)}\label{set-configurations-only-for-edicoes_anteriores-others-are-manually}}

\begin{lstlisting}[language=Python]
# set driver for browser connection
DRIVER_BIN = "/Users/lucas.bruscato/Google Drive/Github-Lucas/master/0_drivers/geckodriver_mac"
# DRIVER_BIN = "C:\\Users\\lbruscato\\Dropbox\\Github-Lucas\\master\\drivers\\geckodriver_windows.exe"

# set initial folder
folder = 'edicoes_anteriores/<folder name here>'
\end{lstlisting}

\hypertarget{capture-search-dates-and-cities-names}{%
\subsubsection{Capture search dates and cities'
names}\label{capture-search-dates-and-cities-names}}

\begin{lstlisting}[language=Python]
# open link file for the chosen folder
link_file = codecs.open(os.getcwd() + "/" + folder + "/link.txt", 'r', "utf-8")
text_link_file = link_file.read().split('\n')

# create the initial and final date to search
initial_date = (text_link_file[1].split('-'))[0]
final_date = (text_link_file[1].split('-'))[1]

# create an empty list of cities
cities = []

# fill cities list
for i in range(2, len(text_link_file)):
    city = text_link_file[i].split('-')
    cities.append(city[1].strip())

print("initial_date = " + initial_date)
print("final_date = " + final_date)
print(cities)
\end{lstlisting}

\hypertarget{open-browser-and-download-all-cities-report}{%
\subsubsection{Open browser and download all cities'
report}\label{open-browser-and-download-all-cities-report}}

\begin{lstlisting}[language=Python]
# set browser preferences and profile (auto download)
profile = FirefoxProfile()
profile.set_preference('browser.helperApps.neverAsk.saveToDisk', "application/pdf,application/zip")
profile.set_preference('browser.download.folderList', 2)
profile.set_preference('browser.download.dir', os.getcwd() + "/" + folder)

# open browser
browser = webdriver.Firefox(executable_path = DRIVER_BIN, firefox_profile = profile)
browser.maximize_window()

for j in range(0, len(cities), 5):
    browser.get('https://auditoria.cgu.gov.br/')
    
    # fill fields
    Select(browser.find_element_by_id("linhaAtuacao")).select_by_visible_text('Fiscalização em Entes Federativos - Municípios')

    browser.find_element_by_id("de").send_keys(initialDate, Keys.TAB)
    browser.find_element_by_id("ate").send_keys(finalDate, Keys.TAB)
    
    for i in range(j, j + 5):
        if (i < len(cities)):
            browser.find_element_by_id("palavraChave").send_keys(cities[i], Keys.COMMAND, 'a')
            browser.find_element_by_id("palavraChave").send_keys(Keys.COMMAND, 'x')
            browser.find_element_by_id("token-input-municipios").send_keys(Keys.COMMAND, 'v')
            time.sleep(2)
            
            browser.find_element_by_id("token-input-municipios").send_keys(Keys.ENTER)
            time.sleep(1)
    
    # search and download files
    browser.find_element_by_id("btnPesquisar").click()
    time.sleep(2)
    
    browser.find_element_by_id("btnSelectAll").click()
    time.sleep(1)
    
    browser.find_element_by_id("btnBaixar").click()
    time.sleep(1)
\end{lstlisting}



% ------------------------------------------------------------------- %
% Bibliografia
\backmatter \singlespacing   % espaçamento simples
\bibliographystyle{plainnat-ime} % citação bibliográfica textual
\bibliography{bibliography}  % associado ao arquivo: 'bibliography.bib'

\end{document}
