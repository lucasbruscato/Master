%% ------------------------------------------------------------------------- %%
\chapter{Contexto}
\label{cap:contexto}

Este trabalho se insere no contexto de coleta de dados estruturados e não estruturados, processamento de linguagem natural (NLP), previsão de polaridade e utilização de informações socioeconômicas.

Serão utilizados os relatórios de auditoria produzidos pelo "Programa de Fiscalização de Entes Federativos", as informações do censo do IBGE de 2000 e 2010 e abordagens de NLP e aprendizado de máquina sobre as informações supracitadas.

\section{Variável Resposta}
\label{sec:variavel_resposta}

Em agosto de 2015 a Controladoria-Geral da União (CGU) iniciou um trabalho que engloba o "Programa de Fiscalização em Entes Federativos", um novo método de controle que está sendo aplicado desde então na avaliação dos recursos públicos federais repassados a estados, municípios e Distrito Federal.

A iniciativa incorporou o antigo Programa de Fiscalização por Sorteios Públicos, sendo que, agora, o programa possui três formas de seleção de entes: Censo, Matriz de Vulnerabilidade e Sorteios. Nesse contexto, já foram fiscalizados cerca de 2,5 mil municípios brasileiros desde 2003, englobando recursos públicos federais superiores ao montante de R\$ 30 bilhões.

Quando é utilizado o Censo, a fiscalização verifica a regularidade da aplicação dos recursos em todos os entes da amostragem. Já a Matriz agrega inteligência da informação, por meio da análise de indicadores, para identificar vulnerabilidades (situações locais críticas) e selecionar de forma analítica os entes a serem fiscalizados em determinada região. A metodologia de Sorteios permanece aleatória, ao incorporar as ações do antigo Programa de Fiscalização por Sorteios Públicos.

No âmbito deste trabalho serão utilizadas os sorteios 34 ao 40, o ciclo 3 e o ciclo 4 (? - talvez ciclo 5).


\section{SentiLex}
\label{sec:sentilex}



\section{Variáveis Explicativas}
\label{sec:variaveis_explicativas}







%% ------------------------------------------------------------------------- %%