%% ------------------------------------------------------------------------- %%
\chapter{Contexto}
\label{cap:contexto}
\section{Variável resposta}
\label{sec:variavel_resposta}

Em agosto de 2015 a Controladoria-Geral da União (CGU) iniciou um trabalho que engloba o "Programa de Fiscalização em Entes Federativos", um novo método de controle que está sendo aplicado desde então na avaliação dos recursos públicos federais repassados a estados, municípios e Distrito Federal.

A iniciativa incorporou o antigo Programa de Fiscalização por Sorteios Públicos, sendo que, agora, o programa possui três formas de seleção de entes: Censo, Matriz de Vulnerabilidade e Sorteios. Nesse contexto, já foram fiscalizados cerca de 2,5 mil municípios brasileiros desde 2003, englobando recursos públicos federais superiores ao montante de R\$ 30 bilhões.

Quando é utilizado o Censo, a fiscalização verifica a regularidade da aplicação dos recursos em todos os entes da amostragem. Já a Matriz agrega inteligência da informação, por meio da análise de indicadores, para identificar vulnerabilidades (situações locais críticas) e selecionar de forma analítica os entes a serem fiscalizados em determinada região. A metodologia de Sorteios permanece aleatória, ao incorporar as ações do antigo Programa de Fiscalização por Sorteios Públicos.


